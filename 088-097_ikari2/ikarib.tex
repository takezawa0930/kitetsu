%\documentclass[10pt,a4j]{utjarticle}
\documentclass[9pt,b5j,twoside,twocolumn]{utarticle}
%\documentclass[b5j,twoside]{utarticle}
%\documentclass[b5j,twoside,twocolumn]{utbook}
\setlength{\columnsep}{2zw}
\usepackage{bxpapersize}
\usepackage{pxrubrica}
\rubysetup{<hj>}
\usepackage{endnotes}
\usepackage{multicol}
\usepackage{plext}
\renewcommand{\theendnote}{[後注\arabic{endnote}]}
\renewcommand{\thefootnote}{\hspace{-0.5mm}\arabic{footnote}}
\usepackage{pxftnright}
\usepackage{fancyhdr}
\setlength{\topmargin}{5mm} % ページ上部余白の設定(182mm x 257mmから計算)。
\addtolength{\topmargin}{-1in} % 初期設定の1インチ分を引いておく。
\setlength{\oddsidemargin}{21mm} % 同、奇数ページ左。
\addtolength{\oddsidemargin}{-1in}
\setlength{\evensidemargin}{17mm} % 同、偶数ページ左。
\addtolength{\evensidemargin}{-1in}
\setlength{\footskip}{-5mm}
%\setlength{\marginparwidth}{23mm}
%\setlength{\marginparsep}{5mm}
\setlength{\textwidth}{225mm} % 文書領域の幅(上下)。縦書と横書でパラメータ(width / height)の向きが変わる。
%\setlength{\textheight}{150mm} % 文書領域の幅(左右)
\makeatletter
\def\@cite#1#2{\rensuji{[{#1\if@tempswa , #2\fi}]}}%%
\def\@biblabel#1{\rensuji{[#1]}}%%%
\makeatother
\usepackage{enumerate}
\usepackage{braket}
\usepackage{url}
\usepackage[dvipdfmx]{graphicx}
\usepackage{float}
\usepackage{amsmath,amssymb}
\newcommand{\relmiddle}[1]{\mathrel{}\middle#1\mathrel{}}
\usepackage{ascmac}
\usepackage{okumacro}
\usepackage{marginnote}
%\usepackage[top=15truemm,bottom=15truemm,left=20truemm,right=20truemm]{geometry}
\usepackage{cleveref}
\usepackage{plext}
\usepackage{pxrubrica}
\usepackage{amsmath}
\usepackage{fancybox}
\usepackage[dvipdfmx]{graphicx}
\usepackage{cancel}
\setcounter{tocdepth}{3}

%\renewcommand{\labelenumi}{(\Alph{enumi})}
\usepackage {scalefnt}
\makeatletter
\@definecounter{yakuchu}
\@addtoreset{yakuchu}{document}% <--- depende on class file
\def\yakuchu{%
\@ifnextchar[\@xfootnote %]
{\stepcounter{yakuchu}%
\protected@xdef\@thefnmark{\theyakuchu}%
\@footnotemark\@footnotetext}}
\def\yakuchutext{%
\@ifnextchar [\@xfootnotenext %]
{\protected@xdef\@thefnmark{\theyakuchu}%
\@footnotetext}}
\def\yakuchumark{%
\@ifnextchar[\@xfootnotemark %]
{\stepcounter{yakuchu}%
\protected@xdef\@thefnmark{\theyakuchu}%
\@footnotemark}}
\makeatother

\usepackage{atbegshi,etoolbox}

\newcounter{newfoot}
\patchcmd{\footnotetext}{\thempfn}{\thenewfoot}{}{}

\newcommand{\evenfootnote}[1]{%
  \ifodd\value{page}%
    \footnotemark%
    \AtBeginShipoutNext{%
      \stepcounter{newfoot}\footnotetext{#1}%
    }%
  \else%
    \stepcounter{newfoot}\footnote{#1}%
  \fi%
}


\pagestyle{fancy}

\title{哲学対話を発生させるロボットシステム------社会インフラへ向けた提案}
\author{五十里翔吾}
\date{\vspace{-5mm}}
\setcounter{page}{88}

\begin{document}
\maketitle

\setlength{\footskip}{-2mm}
\lhead[]{【論考】}
\chead[]{}
\rhead[哲学対話を発生させるロボットシステム------社会インフラへ向けた提案]{}
\lfoot[]{\thepage{}}
\cfoot[]{}
\rfoot[\thepage{}]{}

\let\yakuchu=\endnote
\renewcommand{\footnoterule}{\noindent\rule{100mm}{0.3mm}\vskip2mm}
%\tableofcontents
\thispagestyle{fancy}
\section{はじめに}
本稿の目的は、日常の中に哲学対話を生じさせるシステムを社会実装するための構想を提出することである。さまざまな実践\footnote{日本における哲学プラクティスの歴史は文献\cite{DOKU}などに詳しい。}が主張しているように、日常における哲学対話は、生活にポジティブな影響をもたらすと考えられる。それは、自らの価値観を見直すきっかけとなったり、相互行為における他者への尊重を促したりする。


哲学対話は、主に以下の点でその他の言語コミュニケーションと区別される。
\begin{enumerate}
\setlength{\itemsep}{-2mm} 
\renewcommand{\labelenumi}{\pbox<y>{(\arabic{enumi})}}
\item 参与者間で一つの問いを共有して、それを追求する。
%\item 哲学実践の専門家が、議論を深めるための介入を行う。
\item 「話に途中で割り込まない」「話すときは手を挙げる」、あるいは「専門用語を用いないようにする」などのルールがある。
\end{enumerate}
これらは、哲学対話の場を日常から区別し、より自由な発言を促すために設けられている。\footnote{対話のルールや定義には様々な流儀がある。本稿では講演\cite{TERA}を参考にまとめた。}
%寺田のやつ

これまで、哲学カフェや哲学カウンセリングなど、さまざまな形で哲学対話の実践が行われてきた。現時点では、それらの活動はある日時・場所に、専門家である哲学者が主導する形でイベントが開催されるという運営の形態が主であり、万人にとってアクセスしやすいものではない。しかし、哲学対話は、社会における十全な相互行為の基盤となる価値を提供するものである。よって、対話の機会は、社会のインフラとして整備される必要があると考えられる。


本稿は以上のような関心をもって書かれている。


2章では、グレゴリー・ベイトソンの理論に基づいて、哲学対話によって自己変容がもたらされる過程を記述する。
3章では、2章の分析を踏まえて、哲学対話を発生させるインフラストラクチャーとしてのロボットシステムを提案する。

%インフラとして整備するために必要な仕様を提案する。



\section{哲学対話のシステム論}
哲学対話を発生させるシステムを開発するためには、十全な哲学対話においては何が生起するのかを明らかにする必要がある。
本章の目的は、哲学対話がいかにして人々に影響を与えるのかを記述することである。哲学対話を、人々の生活の中で持続する自己変容のコミュニケーションと捉え、それが成立している状況はどのように描写できるのかをシステム論の立場から検討する。


\subsection{自己変容のコミュニケーション理論}
哲学対話とはどのような相互行為だろうか。それはまず、対話の一種である。それでは、「対話」という語はどのように説明されているのだろうか。


劇作家の平田オリザは、対話と会話を区別して以下のように整理している\footnote{平田オリザ、『対話のレッスン 日本人のためのコミュニケーション術』、講談社、二〇一五}。
\begin{description}
 \setlength{\itemsep}{-2mm} 
\item{「会話」} は互いの細かい事情や来歴を知った者同士のさらなる合意形成に重きを置く、すでに知り合った者同士の楽しいお喋りのことである。
\item{「対話」} は異なる価値観のすり合わせであり、差異から出発したコミュニケーションの往復に重点を置く。
\end{description}
この説明は、用法の説明というよりはむしろ術語の定義である。術語は、特定の前提のもとで主張や議論を行うために概念を整理するものだ。人の相互行為を說明しようと試みると、このようにある種遂行的な描写を行うことになるだろう。
この分類に異論はない。しかし、本章の目的に照らすと、まだ不十分である。なぜなら、本章で検討したいのは、言語を介したある種の相互行為が参与者に対してもたらす持続的な影響だからだ。


%そこで本章では、サイバネティックス理論を応用してシステム論的なコミュニケーション理論を提示したグレゴリー・ベイトソンの議論に従って、
グレゴリー・ベイトソンは、サイバネティックス理論を応用してシステム論的なコミュニケーション理論を提示した。そこで主題となるのは、「ある相互行為が自己というシステムにどのようにして影響を与えるのか」である。\footnote{ところで、マルティン・ブーバーの思想(彼の思想は以下の著書に収められている。マルティン・ブーバー、植田重雄訳、『我と汝・対話』、岩波書店、一九七九)は、「対話の思想」と言われる。ブーバーの関心は「対話が人に及ぼす影響」であった。この点でブーバーのベイトソンと関心を共有している。ブーバーは、人が世界を〈われ―それ〉〈われ―なんじ〉という二通りの方法で「語る」と主張し、二者が「互いに向かい合うこと」を根底において、人のあるべき生を論じた。本章の目的はブーバーの関心と近いが、ブーバーの神学的な議論からは本稿の最終目的である「システムの実装」に関する示唆を引き出すことは容易ではないため、本稿では深くは立ち入らない。}
そこで本章では、ベイトソンの理論を参照し、哲学対話の定式化を試みる。


\subsubsection*{ベイトソンのコミュニケーション理論}
ベイトソンのコミュニケーション理論の核をなすのは、サイバネティックス認識論と論理階型論である。以下、簡単に紹介する。\footnote{文献\cite{JIKO}を参考に、ベイトソンの著書\cite{SEM}を整理した。} \\
\textbf{(1)サイバネティックス認識論}


この認識論によると、一人の生きた人間の、生きた現実にとって「存在すること」と「認識すること」は切り離すことができず(「存在=認識」)、それらはまさに「行為すること」である。\footnote{世界とはこういうものだ(what sort of world it is)という(通常無意識レヴェルの)思い込みが、世界をどのように捉えそのなかでどうふるまうか(how to see and act)ということを決定するわけだし、逆に、かれの知覚と行動のあり方(how)が、世界の何であるか(what)を決定するわけである。(S.E.M. p.450)}行為とは、それ自身が自己を修正する自己言及的な過程である。その過程は円環的な回路として捉えられ、その上を流れる差異は別の差異と交わり、再帰的に参照される。ベイトソンは、この過程を\textbf{精神(mind)}という。\\
\textbf{(2)論理階型論}


論理階型論によると、世界の認知は階層的な学習によって成立している。ゼロ学習は、直接的な刺激の学習である(パブロフの犬)。学習Ⅰは、状況に依存したシグナルを学習する過程である。この学習により、メッセージをコンテクストに照らし合わせて解釈できるようになる。学習Ⅱは、さらなる自己の統合の過程である。「個々の状況を表すパターン」がどのような法則によって生じているのかを学習する(=世界の構造的認知過程、すなわち通常自己と言われるもの)。ベイトソンは、さらに学習Ⅲを定義した。このレベルでは、宗教的「回心」のように、自己そのものが組み替えられる。


世界に働きかけるシステムは、このような階層的な学習を行う。ゼロ学習と学習Ⅰは、我々の個別の\textbf{行為}に関わる。学習Ⅱは、我々が\textbf{自己}と呼ぶ、「その人に染み込んださまざまの前提(S.E.M p.432)」を構成する。そして、学習Ⅲは、「自己」と世界の\textbf{全体}の関わりを規定する。システム論を用いると、この関係は以下のように整理することができる\cite{JIKO}。

(Ⅰ)行為システムは(Ⅱ)自己システムのサブシステムである。そして、自己システムをその上位で(Ⅲ)エコシステムが包摂している。一般システム論によると、部分による部分の修正は不可能である。自己システムの変容は、(a)自己システムの一部がエコシステムからのフィードバックにより変容を受ける、あるいは(b)自己システムにおけるパターンに亀裂が走り、それによってより大きな回路(Ⅲ)における学習Ⅲが生じる(自己の組み換え)という形をとることになる。


これらの学習の過程を理解するには、ベイトソンによる「関係の区別」を導入する必要がある。
ベイトソンは、物事の間の関係を「対称的」なものと「相補的」なものに二分した。
\begin{quote}
二者関係において、AとBの行動が、(AとBによって)同じものとして見られ、しかもAの行動の強まりがBを刺激して同じ(とされる)行動を強め、逆にまたBの行動がAの〝同じ〟行動を促進するようなかたちで二つが連関しているとき、それらの行動に関して両者の関係は「対称的」(symmetric)であるという。


~一方、たとえば見る行為と見せる行為とが互いにフィットするように、AとBの行動が同じではないが相互にフィットするものであり、しかもAの行動の強まりがBの行動の強まりを呼ぶようなかたちで両者が連関しているとき、それらの行動に関して両者の関係は「相補的」(complementary)であるという。(S.E.M. p.462)
\end{quote}


ベイトソンは、(a)の例として芸術鑑賞の経験を挙げている。我々の世界の認識は、無意識下に沈んだ認知枠組みに(例えば遠近法)よって支えられている。芸術作品においては、その無意識のゲシュタルトが主題化されることがある(例えばキュビズム絵画)。そのような作品が伝える無意識のゲシュタルトについての情報は、(Ⅲ)エコシステムの回路に対してフィードバックされる。このようにして、世界の見方が「全体論」的なものに変化し、自己のより高い統合が促される\footnote{ベイトソンは、芸術のこのような作用を「精神を癒やすものとしての芸術」(corrective nature of art)と呼んだ。}。


(b)の例として挙げられているのは、アルコール依存症患者が耽溺から解放されるときの体験である。AA〈アルコホリックス・アノニマス〉におけるアルコールとの戦いは、「十二のステップ」からなる。その最初の二つは以下のようなものである。\footnote{AA日本ゼネラルサービス\rensuji{HP}より。\url{https://aajapan.org/12steps/}}
\begin{enumerate}
\setlength{\itemsep}{-2mm} 
\item 私たちはアルコールに対し無力であり、思い通りに生きていけなくなっていたことを認めた。
\item 自分を超えた大きな力が、私たちを健康な心に戻してくれると信じるようになった。
\end{enumerate}

アルコール依存症患者は、(時に周囲から言われ)誘惑と戦うべきだと信じている。そうして生じる「自らの意思でアルコールに打ち勝つことを示してやる」という酒との「対称的」関係が自己を破滅に向かわせているのだと自覚することが、解放へのはじめの一歩なのだ。この自体は、以下のように説明できる。(Ⅱ)自己システムが「対称的」な関係のもとでは維持できないことが分かったとき、すなわち自己がある種の「絶望」に追いやられたとき、より上位のシステム(Ⅲ)における自己の組み換えが発生する。そして、患者は自己を酒との「相補」的な関係の中に位置づけられるようになる。アルコール依存症患者の「開放=回心\footnote{ベイトソンは、患者の経験する解放を一種の神学的体験と考えていた。}」は、このように説明される。


\subsubsection*{「哲学対話」に向けて}
ベイトソンのいう(a)、(b)の学習による自己変容は、ある種の相互行為を通した自己変容を記述している。これらのどちらの経験も、(Ⅱ)自己システムに学習されたパターンが維持されなくなることで、自己を包摂するより大きなシステムとの「相補的」関係が生じるという事態がその契機であった。
それでは、これらの枠組みは哲学対話の満足な定義を与えるだろうか。もう少し距離があるように考えられる。


(ⅰ)哲学対話を駆動しているのは、何よりもまず、問いとの「対称的」関係である\footnote{いわゆる「哲学混乱」という状態である。}。「相補的」態度\bou{しか}存在しないのであれば、そもそも探究は行われない。一方で、複数人での議論においては、一つのもっともらしい考えが提出されると、皆がそれを受け入れるようになることがある\footnote{社会心理学の理論である「集団極性化」、「フリーライディング」によって説明できる。}。このような現象が生じる背景には、「対称的」なテーマが入り込んでいる。すなわち、「どちらが正しいのだろう」という比較を行う相互行為が前提とされている。ある考えが別の考えに「論破」される場合にそれは顕著である。また、もっともらしい意見を聞いて黙ってしまう、あるいは考えるのをやめるという場合にも、潜在的には、複数の考えを比較してより正しい方を選ぶというディベートの構造がある。異なる意見の間に対話が継続できなくなるほど深刻な対立が生じる、といったケースの背景にも、このような「対称的」な関係がある。



すなわち、「相補的」関係のみでは哲学対話は成立できず、「対称的」関係が支配する場では継続が難しくなる。このことが示すのは、哲学対話においては、「対称的」関係と「相補的」関係が同時に成立しているということだ。


(ⅱ)哲学対話によって生じる自己の変容は、おそらく不可逆的で断絶を伴うものではない。自己に弾力を与え、不確実な状態に対する耐性を高めさせるものである。そこで生じているのは、他者との言語を介した相互行為を経て得られた自らのあり方(「存在=認識」)についての問いを、他者の声をフィードバックさせながら考え追求する、という過程である。%この過程は、どのようにして自己システムの変容をもたらすのだろうか。
すなわち、哲学対話がもたらす自己の変容は、回路に情報が流れることそれ自体(学習Ⅲ)によってではなく、回路間の連絡が持続することによってもたらされる、ということである。


以上を踏まえると、哲学対話の中で生じている相互行為と、その相互行為が人の自己変容を促す過程は以下のように記述できる。


\begin{enumerate}
\renewcommand{\labelenumi}{\pbox<y>{(\arabic{enumi})}}
 \setlength{\itemsep}{-2mm} 
\item 哲学対話において、参与者は①他者(他の参与者)との「相補的」関係+世界との「対称的」関係、あるいは②他者との「対称的」関係+世界との「相補的」関係、のどちらかの状態にあり、それらの間を動的に遷移する。このとき、
「他者」あるいは「世界」のうち(Ⅱ)自己システムと「相補的」関係にあるものが、(Ⅲ)エコシステムから流れ出し、自己システムへと連絡する。しかし、別の一方とは「対称的」関係を結んでおり、その関係が挫折するわけではないので、学習Ⅲによる自己の組み換えは行われない。
\item 哲学対話の中で共有された主題と、「他者」「世界」と自己システムとの、\bou{部分的に「対称的」で部分的に「相補的」}な関係は、
そこで生じた自己システムとエコシステムとの連絡と共に、全体システムの回路に「焼き付け」られる。
\item そして、「対称的」な関係を含む「焼き付け」によって、対話の現場での相互行為が終了した後も、それぞれの参与者が日常の中で主題についての追求を行うことができるようになる。それが新たな哲学対話\pbox<y>{(1)}を誘発する。
\item 同時に、自己システムとエコシステムの連絡が保たれることで、
エコシステムにおいて無意識下に沈んでいた、他者との相互行為において前提とされている認知枠組みや、世界についての無意識のゲシュタルトを意識的に思考の対象にできるようになる。これによって、部分的、表面的ではなく全体論的な世界の見方が獲得され、より統合された自己システムが達成される。
\end{enumerate}


\subsection{自己変容のシステム}
前節では、哲学対話の中で生じている相互行為を記述することを試みた。そこで言われている
①他者(他の参与者)との「相補的」関係+世界との「対称的」関係、②他者との「対称的」関係+世界との「相補的」関係、とはどのような状態で、それらの状態の遷移はどのように生じるのだろうか。


まず、①が表す状況は以下のようなものである。参与者は、自身が認識する世界についてのある問いに対して、少数の理解可能な「原理」によって答えが与えられると想定する。そして、他者と異なる視点や前提に立った、多様な考えを提出するように努める。


一方、②が表す状況は以下のようなものである。参与者は、「今までに出た考えから統一的な答えを得ることは不可能だ」という考えを持ちつつも、他者の考えと自分の考えを比較し、より良い答えを提出しようと自分自身で努める。


個人の中での、これらの状態間の遷移は、哲学対話というより大きなシステムの挙動によって決定される。すなわち、多様な考えが提出されている状態では、②が目指され、意見が少数に統合されようとしている局面では①が目指される。
これらのどちらの状況でも、自己システムの内部はダブル・バインドの状態にある。しかし、哲学対話というシステムはその矛盾によって維持されるのである。


哲学対話システムによって形成される「自己変容の回路」は、一度それが形成されると、現場での相互行為が終了しても、自己システムの上位に保存され、自己システムの統合を促進する。
「自己変容の回路」が形成され、維持されるためには現場での十全な実践が行われ、それが繰り返されることが不可欠である。


%(ⅰ)①、②を満たすな対話が現場で行われ、(ⅱ)その後の生活の中でも活性化される必要がある。すなわち、上記の哲学対話の回路が起動する必要がある。
%(ⅰ)に関しては、哲学カフェなどの実践の場が該当する。(ⅱ)が指すのは、日常生活の中での雑談において哲学な問いが検討される、というような状況である。
しかし、現代社会においてそのような実践の場は限られており、また、哲学対話の場に参加することには難しさが伴う。
%また、そのような場はやはり限られている。
日常生活で哲学的な雑談が行われる機会も、そう多いとは考えにくい。次章では、この問題を解決するため、哲学対話を発生させる、ロボットを用いたインフラとしての対話システムを提案する。

\section{哲学対話ロボットシステム}
%前章では、哲学対話が生じている状態では、どのような相互行為が行われ、参与者のその後における持続的な影響をもたらすかのを記述した。
本章の目的は、哲学対話を生じさせることを目的としたロボット対話システムに必要な仕様を提出することである。





前章で明らかになったことは、哲学対話は生活の中で繰り返し行われる必要があること、そして哲学対話においては、「対称性」と「相補性」が混在する二つの状態を遷移することが重要であることである。


ここで提出するシステムは、これらの要件を満たすものである。
全体システムは三つの部分からなる。
\subsubsection*{(ⅰ)街中で人に問いかける「考えを引き出す自律ロボット」}
このシステムは、前章の①の対話状況を目指すものである。ロボットは駅や学校、遊園地などを徘徊し、道行く人に哲学な問いかけをする。ここでの問いは、後述する(ⅲ)の対話の木から抽出され、人が答えた内容は、そのスタンス(問いに対する答えなのか、何らかの意見に対する賛成あるいは反対なのか、別の形式の意見なのか)と共に、対話の木に登録される。


人は問われることで答えを求めるようになる(問いとの「対称的」関係)と考えられる。また、日常の中で問われることで、その人の生活に根ざした多様な考えが引き出せる(他者との「相補的」関係)。


このシステムにおいてロボットを対話のインターフェースに用いることには、以下のようなメリットがある。
\begin{enumerate}
 \setlength{\itemsep}{-2mm} 
\renewcommand{\labelenumi}{\pbox<y>{(\arabic{enumi})}}
\item 発話速度や音量、また使用する言語やコミュニケーションの形態そのものを、人の労力を要することなく調整でき、より多くの人とのコミュニケーションが可能になる。
\item ロボットを相手にすると人は自己開示を行いやすくなる場合があることが知られており\cite{SHIMA}\cite{KAIJI}、より深い意見を引き出せる可能性がある。
\end{enumerate}

\begin{figure}[h]
\centering
\begin{tabular}<y>{c}
\begin{minipage}[c]{0.60\hsize}
\centering
\includegraphics[scale=0.5]{system1}
\caption{街中にソクラテスとプラトンがいる}
\end{minipage}
\end{tabular}
\end{figure}
ところで、二体のロボットを用いたほうが、ロボットと人のコミュニケーションが自然に行えるという研究がある\cite{MUL}。ゆえに、このシステムは、複数のロボットを連携させて人に問いかけを行うことで、より意見を引き出しやすくなると考えられる。(二体のロボットの名前は「ソクラテス」と「プラトン」としてはどうだろうか。)\\
\textbf{(ⅱ)ロボットに仲介された「タッチパネル越しの対話システム」}\\
このシステムは、前章の②の対話状況を目指すものである。


このシステムは、三人程度で使用するシステムである。駅や公園などの公共の場に設置されることを想定している。
対話の場には、ロボットが参与する人の数だけ設置されている。人はタッチパネルを持ち、自分のロボットに発話させる内容を表示された選択肢の中から選ぶ。
発話内容を選ぶことで、対話の木が(ⅲ)展開されていく。このとき、対話の場にいる人が過去に(ⅰ)のシステムを通して登録した意見が優先的に発話の選択肢に選ばれるようにする。
すなわち、対話の参与者は、自分がかつて発した考えや、その発話に関連する主張を、自分あるいは他の参与者が操作するロボットの発話を通して聞くことになる。
%ここで表示される選択肢は、対話のシナリオに沿ったものになる。
%対話のシナリオは、対話の場にいる人が過去に(ⅰ)のシステムを通して登録した意見が発話されるように(ⅲ)対話の木を展開していくという形式をとる。すなわち、対話の参与者は、自分がかつて発した考えや、その発話に関連する主張を、自分あるいは他の参与者が操作するロボットの発話を通して聞くことになる。%この対話の場では、他者が普段の生活の中で考えた多様な意見に触れることになる。よって、それらを比較して深めるという形式(他者との「対称的」関係)の対話が促され、対話の中で局所的な結論が目指されることが少なくなる(問いとの「相補的」関係)と考えられる。


タッチパネルのインターフェースは、以下のような選択肢を表示する。
%\begin{enumerate}
%\renewcommand{\labelenumi}{\pbox<y>{〈\arabic{enumi}〉}}
\pbox<y>{〈1〉}対話の木を展開することで得られる登録された内容を発話する。


木を展開する規則は以下のようなものである。木の浅い部分から、それまでに展開されたノード(意見)の子ノード(上の意見に対する賛成、反対、補足など)の中からランダムにいくつかが選ばれる。このとき、参与者の誰かによって登録されたノードを優先的に選んでいく。
また、選ぶ選択肢がない場合には、ランダムにノードを遷移するか、参与者のだれかによって木の浅い部分に登録された意見を選ぶ。このとき、なるべく浅いノードで、子ノードに参与者が登録したノードが多く存在するものを選ぶ。\\\pbox<y>{〈2〉}あいづち、フィラーを発する。


「なるほど」や「たしかに」といった、対話を自然にする「あいづち」集が事前に用意されており、そこからランダムに選ばれて表示される。\\
\pbox<y>{〈3〉}口頭で発言する。


ある参与者がこの選択肢を選ぶと、選んだ人に向けてその場にいるロボットが発話を促す。対話が進むにつれて、そこで出た考えを踏まえて新たな意見が形成される可能性がある。そのような場合に、参与者はこの選択肢を選んで発言することができる。
そして、発せられた意見は「哲学対話の木」に登録される。\footnote{このプロセスにおいては、重複した意見が登録されないようにしたり、明らかな誹謗などを除くような仕組みを整備する必要がある。}
%\end{enumerate}

このシステムを介した会話は、以下のような理由から、社会における
%このシステムは、対話に以下のような影響をもたらすことで、
より深い哲学対話を促進することができると考えられる。\footnote{以下は、すべて検証が必要な仮説である。筆者は卒論などを通じてこれらの仮説を詳しく検討するつもりである︙︙}\\
(a)このシステムを通じた対話においては、事前に登録された意見が、対話の木が展開される過程で共有される。つまり、「考える」プロセスと「相手に伝える」プロセスが分離された状態で対話が行われる。これによって「相手に考えながら伝える」あるいは「相手に伝えながら考える」ことの難しさが緩和され、対話自体に参加しやすくなる。\\
(b)参与者全員が容易に発話できることで、誰か一人が、意見の共有を超えて「相手を説得する」発話を行うことが難しくなる。そのため、それぞれが普段の生活の中で形成された多様な考えが自然に共有される。その結果、それぞれの考えをより深く知ろうという気持ちが働く。\\
(c)また、意見の共有がロボットを通して行われることで、それらから距離をおいて検討することができる。特に、自分が以前に考えたことがロボットを通して語られることで、自分の意見を客観視することができるようになる。\footnote{この効果を達成するためには、もう少し作り込みが必要かもしれない。例えば、自分が登録した意見は自分で選択することができないようにする、など。}\\
%また、意見の共有がロボット同士の会話という形式で行われることで、
(d)以上により、様々な意見を比較して深めるという姿勢(他者との「対称的」関係)が生まれ、対話の中で局所的な結論が目指されにくくなる(問いとの「相補的」関係)と考えられる。\\%同時に、それぞれの考えをより深く知ろうという気持ちが働き、対話の意欲が高まると考えられる。
(e)さらに、参与者は対話の木が展開されていく過程で、その場にいない他者が考えた意見にも触れることになる。対話の中で多様な意見に触れることで、新たな視点が生まれやすくなる。\\
(f)ある参与者が「口頭で発言する」という選択肢を選ぶと、ロボットに促された後にその参与者は口頭で考えを述べることになる。
%選択肢にはない意見を誰かが思いついたときには、ロボットに促されることで口頭で発言することができる。この機能により、
これを契機として、次第に口頭での対話に移行することが期待できる。そのときには、すでに対話の流れが共有されているため\footnote{この効果を達成するために、対話の初期には「口頭で発言する」という選択肢を選べないように設計する必要がありそうだ。}対話に齟齬が生じにくいと考えられる。よって口頭での対話において、それぞれの考えがタッチパネル越しの対話の中で生じた新たな視点を通して、深く検討される。\\

%他の参与者はロボットを通じた対話を継続することもできるし、完全に口頭のコミュニケーションに移行することもできる。新たな考えが生じたり、互いの考えを知ろうという気持ちが高まったりすると、次第に口頭での対話に移行すると考えられる。口頭での対話に移行した後も、すでに対話の流れが共有されているため、対話に齟齬が生じにくく、それぞれの考えが深く検討されやすい。
システムを通じた対話の中で、参与者は自分や他の参与者が操作するロボットを\bou{通じて}他者と\bou{出会い}、また、自らと\bou{再会する}。その過程の中で、自らが生きる世界についての「統一的な見解」が存在することの難しさを知り、同時に自らの価値観を見直すきっかけを得るのである。


このシステムは、友達同士で普段しないような話をしたいとき、職場の人間同士で相互理解を深めたいとき、またなんとなく誰かと深い話をしたいとき、などに利用することができる。問いを共有してそれぞれの考えを知ることができる機会を提供するシステムには、ある程度の需要があるのではないか。\\
\textbf{(ⅲ)問いに対する人々の考えを構造化する「哲学対話の木」}\\
哲学対話の木は、さまざまなトピックについての人々の考えの連関を木構造で整理したものである。
%この対話の木を通して、上記の(ⅰ)、(ⅱ)のシステムを通した対話のファシリテーションが間接的に行われる。


まず、大元となる木は、哲学対話の専門家によって、それぞれのトピックに関連する既存の議論を踏まえて構成される。
(ⅰ)、(ⅱ)のシステムがどのトピックを選択するかは場所や時期によって変化させる。例えば、今週の大阪は「死」で、来週のロンドンでは「友達」というトピックが選ばれる、といった具合に。\footnote{このようなシステムが存在するということが社会で共有されること自体によって、(ⅰ)、(ⅱ)のインターフェースを介さずとも、日々の会話の中で「問いを共有した対話」が発生しやすくなるという可能性もある。}
%すなわち、複数の哲学対話の木が社会の中で自己組織化する。



専門家たちは(ⅰ)の自律ロボットを通じて登録された発話に対して、さらに問いかけたり別の観点を提示する発話を登録することで、「哲学対話の木」を発展させる。%\footnote{もちろん、最終的には自動化を行いたいが、現時点でそれが可能だとは思えない。}
上で述べたとおり、(ⅱ)のシステム上で行われる対話は、このようにして組織化される対話の木を展開することによって構成される。
ゆえに、個々の現場には専門家が介在しなくとも、間接的なファシリテーションが行われることになる。\\
%このようにして組織化される哲学対話の木は、それが展開されることで、%(ⅱ)のシステム上で行われる対話を方向づける。
%このようにして組織化される哲学対話の木を展開するという方式で(ⅱ)の対話システムのシナリオが構成されることで、(ⅱ)のシステムを介して
%を構成することで、個々の現場に専門家が居なくても、専門知の恩恵を受けることができるようになる。


%哲学対話が発生しやすくなるという可能性もある。\\
\begin{figure}[h]
\centering
\begin{tabular}<y>{c}
\begin{minipage}[c]{0.68\hsize}
\centering
\includegraphics[scale=0.39]{system3}
\caption{愛についての「哲学対話の木」のイメージ}
\end{minipage}
\end{tabular}
\end{figure}


以上の三つのシステムによって構成される哲学対話ロボットシステムを、社会における活発な哲学実践を醸成するためのインフラとして提案する。
\begin{figure}[h]
\centering
\begin{tabular}<y>{c}
\centering
\begin{minipage}[c]{0.8\hsize}
\centering
\includegraphics[scale=0.55]{system2}
\caption{システムの全体像}
\end{minipage}
\end{tabular}
\end{figure}


(ⅰ)「考えを引き出す自律ロボット」と(ⅱ)「タッチパネル越しの対話システム」は、(ⅲ)「哲学対話の木」によって繋がっている。%人々が日々の生活の中で行った語りが、実践の専門家の手によって、哲学な対話の構造の中に配置されていく。

「考えを引き出す自律ロボット」が人々の生活に問いを生じさせる。そして、人々が日々の生活の中で行った「問いに対する語り」は、実践の専門家の手によって、対話の構造の中に配置されていく。このように構成される「哲学対話の木」を介して、人々の思考は「タッチパネル越しの対話」\footnote{筆者は、タッチパネルというインターフェースが、本稿の目標とする実装に最適な唯一の答えであるとは考えていない。しかし、現代社会におけるスマートフォンやタブレット、またそれを介したコミュニケーション・ツールの流行を見るに、人々が求める直感に、ある程度は適合しているのだと思う。}の場で共有される。人々の思考は、対話の中で他者の語りと出会い、他の様々な思考と比較され、新たな問いを生む。そして、この過程で生じる思索が、それぞれの世界観をより豊かに形作っていくのである。


%もちろん、現在の技術では、特に(ⅰ)など、実装が難しいものもある。本稿では立ち入った検討は行えないが、今後は哲学プラクティス、ロボット工学、自然言語処理などの研究者が連携することで、技術を発展させていく必要があるだろう。
%もちろん、現在の技術では、特に(ⅰ)など、実装が難しいものもある。%小中学校や介護施設などのコミュニティをフィールドに技術開発と実証研究を


\section{おわりに------対話ロボットの未来}
本稿では、哲学対話が成立する場面で、どのようにして自己変容がもたらされるのかを、ベイトソンのコミュニケーション理論に依拠して記述した。そして、その記述によって明らかになった要求を満たす哲学対話システムを提案した。このシステムは、インフラとしての社会実装を目指すものである。


この対話システムには、ロボットというインターフェースが用いられる。このあり方は、今後社会に実装される対話ロボットの一つの未来を示しているようにも考えられる。本稿の結びとして、この点について簡単に触れる。人に理解できる言語を発する対話ロボット\footnote{認知科学者、ロボット工学者の岡田美智男が手掛ける「むー」のように、人と同じ言語を発することをそもそも目指していないコミュニケーション・ロボットも存在する。弱いロボット:「自らの弱さや不完結さを適度に開示しつつ、周囲の人からの手助けを上手に引き出しながら、一緒になって合目的的な行為を組織していくロボット」\url{https://www.icd.cs.tut.ac.jp/}}は、人に似せようとして設計されてきた。しかし、人とロボットは生理的特徴を共有していないし、同じ社会的文脈に置かれることもない。そして、電源を切ると無残にもうなだれてしまう。一方で、多くの日常会話は、互いの立場や職業などに依存するか、生活の中で生じる経験に立脚して発展する。Pepperくんに話しかけたときに感じる虚しさは、このようなズレに起因するのではないだろうか。だとすると、技術が進歩しても、この溝はなかなか埋まるものではないのではないか。


哲学対話においては、可能な限り、習慣や日常的な拘束力はその場から排除されるように努められる。そこでなら、対話ロボットはまたとないパートナーになれるかもしれない。かれら(あるいはそれら)は我々の生活の外側からやってきて、日常の様々な「あたりまえ」を疑うように迫るのだ。


\vspace{-3mm}
\begin{thebibliography}{99}
{\small
\bibitem{DOKU}鷲田清一(監修)『ドキュメント臨床哲学』、シリーズ臨床哲学1、大阪大学出版会、二〇一〇
\bibitem{TERA}寺田 俊郎、平成\rensuji{28}年度倫理学専攻講演会講演要旨 哲学対話の可能性 (原田覺教授 塩谷政憲教授退職記念)、国士舘大学哲学会、国士館哲学、二〇一七、二一、一八〜二六頁
\bibitem{JIKO}亀山佳明「自己変容のコミュニケーション------G・ベイトソン・ノート--」香川大学一般教育研究、ニ〇一二、\rensuji{33}巻 二三七〜二五七頁
\bibitem{SEM}[S.E.M.] Gregory Bateson. Step to an Ecology of Mind, \textsl{University of Chicago press,} 1972 (republished 2000).(グレゴリー・ベイトソン、佐藤良明・高橋和久訳『精神の生態学〔上・下〕』、思索社、一九八七)
\bibitem{KAIJI}内田 貴久、高橋 英之、伴 碧、島谷 二郎、吉川 雄一郎、石黒 浩 (2017)\\ロボットによる傾聴を通じた自己開示の促進、\footnotesize 日本認知科学会第34回大会 \small
\bibitem{SHIMA}\pbox<z>{Kumazaki H, Warren Z, Swanson A, Yoshikawa Y, Matsumoto Y, Tak}\\\pbox<z>{ahashi H, Sarkar N, Ishiguro H, Mimura M, Minabe Y and Kikuchi M}\\\pbox<z>{ (2018) Can Robotic Systems Promote Self-Disclosure in Adolescents}\\\pbox<z>{ with Autism Spectrum Disorder? A Pilot Study. Front. Psychiatry}\\\pbox<z>{ 9:36. doi: 10.3389/fpsyt.2018.00036}
\bibitem{MUL}\pbox<z>{Arimoto, T., Yoshikawa, Y. \& Ishiguro, H. (2018) Multiple---Robot Co}\\\pbox<z>{nversational Patterns for Concealing Incoherent Responses. Internation}\\\pbox<z>{al Journal of Social Robotics. 10. doi: 10.1007/s12369-018-0468-5}

}
\end{thebibliography}
%対話ロボットのあり方なのでは?

%哲学はどこからでも始まるので、このシステムを日常会話とどうつなぐか



\end{document}
