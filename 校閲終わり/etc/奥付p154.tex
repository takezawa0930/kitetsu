%\documentclass[10pt,a4j]{utjarticle}
\documentclass[b5j,twoside]{utarticle}
%\documentclass[b5j,twoside]{utarticle}
%\documentclass[b5j,twoside,twocolumn]{utbook}
\setlength{\columnsep}{2zw}
\usepackage{bxpapersize}
\usepackage{pxrubrica}
\rubysetup{<hj>}
\usepackage{color}
\usepackage{endnotes}
\usepackage{multicol}
\usepackage{plext}
\renewcommand{\theendnote}{[後注\arabic{endnote}]}
\renewcommand{\thefootnote}{\arabic{footnote}}
\usepackage{pxftnright}
\usepackage{fancyhdr}
\setlength{\topmargin}{5mm} % ページ上部余白の設定(182mm x 257mmから計算)。
\addtolength{\topmargin}{-1in} % 初期設定の1インチ分を引いておく。
%\setlength{\oddsidemargin}{14mm} % 同、奇数ページ左。
\addtolength{\oddsidemargin}{-1in}
\setlength{\evensidemargin}{-20mm} % 同、偶数ページ左。
%\addtolength{\evensidemargin}{-1in}
\setlength{\footskip}{-5mm}
%\setlength{\marginparwidth}{23mm}
%\setlength{\marginparsep}{5mm}
%\setlength{\textwidth}{225mm} % 文書領域の幅(上下)。縦書と横書でパラメータ(width / height)の向きが変わる。
\setlength{\textheight}{165mm} % 文書領域の幅(左右)


\makeatletter
\def\@cite#1#2{\rensuji{[{#1\if@tempswa , #2\fi}]}}%%
\def\@biblabel#1{\rensuji{[#1]}}%%%
\makeatother
\usepackage{enumerate}
\usepackage{braket}
\usepackage{url}
\usepackage[dvipdfmx]{graphicx}
\usepackage{float}
\usepackage{amsmath,amssymb}
\newcommand{\relmiddle}[1]{\mathrel{}\middle#1\mathrel{}}
\usepackage{ascmac}
\usepackage{okumacro}
\usepackage{marginnote}
%\usepackage[top=15truemm,bottom=15truemm,left=20truemm,right=20truemm]{geometry}
\usepackage{cleveref}
\usepackage{plext}
\usepackage{pxrubrica}
\usepackage{amsmath}
\usepackage{fancybox}
\usepackage[dvipdfmx]{graphicx}
\usepackage{cancel}
\setcounter{tocdepth}{3}

%\renewcommand{\labelenumi}{(\Alph{enumi})}
\usepackage {scalefnt}
\makeatletter
\@definecounter{yakuchu}
\@addtoreset{yakuchu}{document}% <--- depende on class file
\def\yakuchu{%
\@ifnextchar[\@xfootnote %]
{\stepcounter{yakuchu}%
\protected@xdef\@thefnmark{\theyakuchu}%
\@footnotemark\@footnotetext}}
\def\yakuchutext{%
\@ifnextchar [\@xfootnotenext %]
{\protected@xdef\@thefnmark{\theyakuchu}%
\@footnotetext}}
\def\yakuchumark{%
\@ifnextchar[\@xfootnotemark %]
{\stepcounter{yakuchu}%
\protected@xdef\@thefnmark{\theyakuchu}%
\@footnotemark}}
\makeatother

\usepackage{atbegshi,etoolbox}

\newcounter{newfoot}
\patchcmd{\footnotetext}{\thempfn}{\thenewfoot}{}{}

\newcommand{\evenfootnote}[1]{%
  \ifodd\value{page}%
    \footnotemark%
    \AtBeginShipoutNext{%
      \stepcounter{newfoot}\footnotetext{#1}%
    }%
  \else%
    \stepcounter{newfoot}\footnote{#1}%
  \fi%
}

\newcommand{\shokai}[4]{%
\raggedright\large\textbf{#1}\\%
\footnotesize\raggedleft #2\\%
#3\\
\small\raggedright #4\\%
\vspace{2mm}
}
\newcommand{\shokaimnasi}[3]{%
\raggedright\large\textbf{#1}\\%
\footnotesize\raggedleft #2\\%
\small\raggedright #3\\%
\vspace{2mm}
}

\renewcommand{\baselinestretch}{0.9}
\pagestyle{fancy}

\setcounter{page}{154}

\begin{document}

\setlength{\footskip}{-2mm}
\lhead[]{}
\chead[]{}
\rhead[希哲会について]{}
\lfoot[]{}
\cfoot[]{}
\rfoot[]{}
\thispagestyle{fancy}
\begin{multicols}{4}
\subsection*{執筆者紹介}
\shokai{五十里翔吾}{基礎工学部システム科学科\rensuji{B4}}{anchor200km@gmail.com}{今号からは会長を引き継いで編集を担当しました。立体ミネルヴァくんもぜひ組み立ててみてください。}
\shokai{野上貴裕}{文学部哲学・思想文化学専修\rensuji{B4}}{takahiro.nogami729@gmail.com}{なんとかかんとか出した希哲の1号、2号でしたが、編集を下の世代に託したこの3号以降もプラットフォームとして続いていってくれれば幸せですね。}
\shokai{澤井優花}{文学部哲学・思想文化学専修\rensuji{B2}}{Y00viola@gmail.com}{私自身、しんのすけとよく似ていると思います。}
\shokai{森川勇大}{大学院人間科学研究科\rensuji{M1}}{marumo3da@gmail.com}{さいきん果物にハマっています。}
\shokaimnasi{黒臺瞭太}{阪大文学部\rensuji{OB}}{また呼んでいただいてありがたい限りです。最近アイデアが湧いてまた小説が書きたくなってきました。「希哲」の創刊号で書いて大爆死した(主観)くせに懲りないもんです}
\color{white}
。\\
\color{black}
\shokai{武澤里映}{文学部美学専修\rensuji{B3}}{tkzw0930@gmail.com}{生みの苦しみを味わいましたが、確かにクセになりますね。}
\shokaimnasi{新井悠介}{経済学部\rensuji{B4}}{二度目まして、はじめてにもまして大変なことにね。}
\shokai{佐原キオ}{a}{a}{a}
\shokai{中野由梨花}{法学部法学科\rensuji{B4}}{1996lilyflower@gmail.com}{多分、白くて丸いものが好きです。特に楕円形。}
\shokai{金重}{a}{a}{a}
\shokai{岩本智孝}{大学院文学研究科\rensuji{M1}}{ponrock288okj75@gmail.com}{「大阪梅田」も「京都河原町」もなんだかんだ慣れていくでしょう。ただし「石橋阪大前」、テメーはダメだ。}
\shokai{中谷拓也}{文学部\rensuji{B1}}{tooker1130@gmail.com}{ぼくの座右の銘(勝手に改変済)を聞いてください!「明日にりんごの樹を植える」}
\shokaimnasi{田所穐來}{\rensuji{JK}アルバイター}{エモーショナルハードコア}
\shokaimnasi{小山詩乃}{人間科学部\rensuji{B1}}{精進したいです。}

\color{white}
 . . . . . . . . . . . . . . . . . . . . . . . . . . . . . . . . . . . . . . . . . . . . . . . . . . . . . . . . . . . . . . . . . . . . . . . . . . . . . . . . . . . . . . . . . . . . . . . . . . . . . . . . . . . . . . . . . . . . . . . . . . . . . . . . . . . . . . . . . . . . . . . . . . . . . . . . . . . . . . . . . . . . . . . . . . . . . . . . . . . . . . . . . . . . . . . . . . . . . . . . . . . . . . . . . . . . . . . . . . . . . . . . . . . . . . . . . . . . . . . . . . . . . . . . . . . . . . . . . . . . . . . . . . . . . . . . . . . . . . . . . . . . . . . . . . . . . . . . . . . . . . . . . . . . . . . . . . . . . . . . . . . . . . . . . . . . . . . . . . . . . . . . . . . . . . . . . . . . . . . . . . . . . . . . . . . . . . . . . . . . . . . . . . . . . . . . . . . . . . . . . . . . . . . . . . . . . . . . . . . . . . . . . . . . . . . . . . . . . . . . . . . . . . . . . . . . . . . . . . . . . . . . . . . . . . . . . . . . . . . . . . . . . . . . . . . . . . . . . . . . . . . . . . . . . . . . . . . . . . . . . . . . . . . . . . . . . . . . . . . . . . . . . . . . . . . . . . . . . . . . . . . . . . . . . . . . . . . . . . . . . . . . . . . . . . . . . . . . . . . . . . . . . . . . . . . . . . . . . . . . . . . . . . . . . . . . . . . . . . . . . . . . . . . . . . . . . . . . . . . . . . . . . . . . . . . . . . . . . . . . . . . . . . . . . . . . . . . . . . . . . . . . . . . . . . . . . . . . . . . . . . . . . . . . . . . . . . . . . . . . . . . . . . . . . . . . . . . . . . . . . . . . . . . . . . . . . . . . . . . . . . . . . . . . . . . . . . . . . . . . . . . . . . . . . . . . . . . . . . . . . . . . . . . . . . . . . . . . . . . . . . . . . . . . . . .
\end{multicols}
\end{document}
