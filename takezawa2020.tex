%\documentclass[10pt,a4j]{utjarticle}
\documentclass[b5j,twoside,twocolumn]{utarticle}
%\documentclass[b5j,twoside]{utarticle}
%\documentclass[b5j,twoside,twocolumn]{utbook}
\setlength{\columnsep}{2zw}
\usepackage{bxpapersize}
\usepackage{pxrubrica}
\rubysetup{<hj>}
\usepackage{endnotes}
\usepackage{multicol}
\usepackage{plext}
\renewcommand{\theendnote}{[後注\arabic{endnote}]}
\renewcommand{\thefootnote}{\arabic{footnote}}
\usepackage{pxftnright}
\usepackage{fancyhdr}
\setlength{\topmargin}{5mm} % ページ上部余白の設定(182mm x 257mmから計算)。
\addtolength{\topmargin}{-1in} % 初期設定の1インチ分を引いておく。
\setlength{\oddsidemargin}{21mm} % 同、奇数ページ左。
\addtolength{\oddsidemargin}{-1in}
\setlength{\evensidemargin}{17mm} % 同、偶数ページ左。
\addtolength{\evensidemargin}{-1in}
\setlength{\footskip}{-5mm}
%\setlength{\marginparwidth}{23mm}
%\setlength{\marginparsep}{5mm}
\setlength{\textwidth}{225mm} % 文書領域の幅(上下)。縦書と横書でパラメータ(width / height)の向きが変わる。
%\setlength{\textheight}{150mm} % 文書領域の幅(左右)
\makeatletter
\def\@cite#1#2{\rensuji{[{#1\if@tempswa , #2\fi}]}}%%
\def\@biblabel#1{\rensuji{[#1]}}%%%
\makeatother
\usepackage{enumerate}
\usepackage{braket}
\usepackage{url}
\usepackage[dvipdfmx]{graphicx}
\usepackage{float}
\usepackage{amsmath,amssymb}
\newcommand{\relmiddle}[1]{\mathrel{}\middle#1\mathrel{}}
\usepackage{ascmac}
\usepackage{okumacro}
\usepackage{marginnote}
%\usepackage[top=15truemm,bottom=15truemm,left=20truemm,right=20truemm]{geometry}
\usepackage{cleveref}
\usepackage{plext}
\usepackage{pxrubrica}
\usepackage{amsmath}
\usepackage{fancybox}
\usepackage[dvipdfmx]{graphicx}
\usepackage{cancel}
\setcounter{tocdepth}{3}

%\renewcommand{\labelenumi}{(\Alph{enumi})}
\usepackage {scalefnt}
\makeatletter
\@definecounter{yakuchu}
\@addtoreset{yakuchu}{document}% <--- depende on class file
\def\yakuchu{%
\@ifnextchar[\@xfootnote %]
{\stepcounter{yakuchu}%
\protected@xdef\@thefnmark{\theyakuchu}%
\@footnotemark\@footnotetext}}
\def\yakuchutext{%
\@ifnextchar [\@xfootnotenext %]
{\protected@xdef\@thefnmark{\theyakuchu}%
\@footnotetext}}
\def\yakuchumark{%
\@ifnextchar[\@xfootnotemark %]
{\stepcounter{yakuchu}%
\protected@xdef\@thefnmark{\theyakuchu}%
\@footnotemark}}
\makeatother

\usepackage{atbegshi,etoolbox}

\newcounter{newfoot}
\patchcmd{\footnotetext}{\thempfn}{\thenewfoot}{}{}

\newcommand{\evenfootnote}[1]{%
  \ifodd\value{page}%
    \footnotemark%
    \AtBeginShipoutNext{%
      \stepcounter{newfoot}\footnotetext{#1}%
    }%
  \else%
    \stepcounter{newfoot}\footnote{#1}%
  \fi%
}


\pagestyle{fancy}

\title{美的性質としてのしらふ}
\author{武澤里映}
\date{\vspace{-5mm}}
\setcounter{page}{101}

\begin{document}
\maketitle

\setlength{\footskip}{-2mm}
\lhead[]{【論考】}
\chead[]{}
\rhead[美的性質としてのしらふ]{}
\lfoot[]{\thepage{}}
\cfoot[]{}
\rfoot[\thepage{}]{}

\let\yakuchu=\endnote
\renewcommand{\footnoterule}{\noindent\rule{100mm}{0.3mm}\vskip2mm}
%\tableofcontents
\thispagestyle{fancy}
\section*{はじめに}
しらふ\footnote{しらふという言葉は、アルコール依存症患者などの依存症の治療の場合にも用いられるが、本稿は飲み会など日常生活の中で使われる意味のしらふを対象とする。例えば「しらふだととても言えない」や、「しらふじゃなかったっけ」などの用法で用いられるような場合のしらふを扱う。}とはどのような性質なのだろうか。本稿では、この問いに対し「しらふとは美的性質である」という提案を行いたい。そのために、第1節ではしらふという性質は、飲酒していないという身体状態以外にも、①その対象がお酒に関わる環境に属していることについての社会的、慣習的な知識、②ゲシュタルト的な知覚、③判断や解釈とも異なるしらふという性質の知覚という3点が必要とされることを導き出す。そして第2節で、上記3点を説明できる性質として美的性質を提示し、しらふが美的性質と言えることを示す。

しらふは対象の身体状態に単に言及するようにみえる一方で、対象をしらふだと判断することは、単に相手の身体状態を指示する以上の価値づけをしているように思われる。例えば、飲酒していないのに飲酒している人と同様の態度を見せている人に対し、「しらふだよね?」などという言い方でしらふであることを指摘する場面があるが、ここでのしらふは身体状態だけではなく、落ち着いている、冷静であるといった主観的な価値づけをも意味しているように思われる。このようなしらふは、一方的な価値づけによる不快感を引き起こすことがあるが、お酒を飲ませようと勧める場合とは異なり道徳的な善悪によってその判断の是非を調べることができない。

しらふを美的性質だとすることで、このようなしらふの倫理性を新たな視点から論じることができる。美的性質を人間に付与することは、かわいいや美しいなどの美的性質に顕著であるように、しばしば倫理性が問題になる。また、しらふという、かわいいや美しいと比べるとその性質の内容の是非は問題になりにくい性質を美的性質とすることで、美的性質の知覚および判断の倫理性についても問うことができる。\footnote{このように美的性質を付与する行為自体の倫理性を考えることは、西村(2011)による「美的フレーミング」の倫理性に関する議論の延長とも言うことができる。「美的フレーミング」とは、「対象についての非美的知覚を一定のコンセプトに基づいた特定の条件枠のもとで美的に組織立て構造化する社会的、文化的、慣習的なふるまい」(西村2011:58)だとされるが、西村はプラスチックの木などを例にして、そのフレーミング自体の倫理性を吟味することが必要だとしている。(ibid.: chap5)}

第1節に入る前に、まず本稿でしらふをどのような美的性質として扱うかを明確化したい。美的性質の例としてしばしば挙げられるのは「美しい」「優美だ」などだが、例えば「ダイナミックだ」「鮮やかな」などの性質も美的性質に含まれる。美的概念 について論じた分析美学者フランク・シブリ―は美的用語(美的性質)を以下の3つに分類している。1つ目が、「よい」「醜い」などの「評価のみの用語」と呼ばれる。2つ目は、かみそりに対する「鋭さ」や、テニスボールに対する「丸さ」などの記述的でありかつ特定の価値を記述する用語で、「記述的な価値用語」と呼ばれる。3つ目が、「おいしい」などの「評価付加的な用語」という記述的でありかつ評価的な性質である。(Sibely1974:5-6)

本稿は、しらふという性質を、上記のシブリ―の分類のうち2つ目の「記述的な価値用語」の1つとして扱う。理由は、まずしらふという用語は、対象に特定の性質を記述していると考えられること、そして「記述的な価値用語」が対象が何であるかにある程度依存しうるように、しらふもまた対象がどのようなカテゴリーだと理解されるかに依存していると考えられることからである。本稿では、このような対象のカテゴリーを確定するものとして、しらふの知覚における知識の必要性を説くことになる。


\section{第1節}
\subsection{1-1.しらふは内在的性質か?}
しらふとはまず「酒気を帯びていないこと」という意味を持つ。\footnote{「デジタル大辞泉」、「コトバンク」https://kotobank.jp/word/%E7%B4%A0%E9%9D%A2-535589、2020年8月9日閲覧。}
これに従えば、例えば血液検査や口臭検査をすればある人がしらふかどうかは判断することができる。しかし本当に、しらふという性質はある人の身体のみを検査することでわかる性質なのか。

確かに、しらふがアルコールの入っていない身体状態と考えれば、しらふな人と飲酒している人を区別することができる。その意味で、しらふは質量や形などと同じく対象に内在する性質だろう。しかし、これはしらふという性質を説明するのに十分とは言えない。というのは、体内にアルコールがないという状態は、日々の大半をその状態で過ごすような自然な状態でもあるからだ。日常の中での飲酒していない状態は、しらふではなくむしろ普段通りや普通などと呼ばれるだろう。つまり、対象がしらふだと知覚されるのは、飲み会などの特別な状況のみである。よって、しらふは単にアルコールがないという身体の状態のみならず、対象が飲み会などのアルコールがある環境に属していることが必要になる。ここから、しらふを知覚する際には、対象の飲酒していないという身体状態に加え、①対象がお酒に関する環境に属していること、そしてそれを知覚する主体が、対象がお酒に関係する環境に属していると知っていることが必要だと言える。

\subsection{1ー2.しらふは傾向的性質か?}

1ー1によって、しらふは何らかの環境に対象が属して初めて浮かび上がるような内在的性質だということが明らかになった。一般に、ある特定の環境や出来事でもってはじめて顕在化する性質は、傾向的性質と呼ばれる。例えば水溶性や怒りっぽさなどが傾向的性質であるが、しらふもまた傾向的性質といえるのだろうか。柏端の定義に従えば、傾向的性質は以下のように表される。

\begin{quote}
物xは傾向的性質Dをもつ⇔df.もし適切な状況Cのもとで出来事Tが生起したならば、xに出来事Mが生起する(柏端2017:168)
\end{quote}

これに従えば、例えば砂糖の水溶性は、砂糖が溶けるのに適切な水があったうえで、砂糖を水に沈めるという出来事が生起したならば、砂糖が溶けるという形で水溶性が顕在化すると説明できる。しらふもまた、前節で指摘したように、飲み会などという適切な環境のもとで、飲酒しないという出来事がきっかけになってしらふであることが顕在化することになる。

しかし一方で、しらふはこの傾向的性質の定義とは相いれない特徴もある。それは、しらふには出来事Mが存在しないという点である。砂糖の水溶性は砂糖が水に溶け形が変わることによって確認できる。しかし、しらふの場合は、飲酒していない普通の状態からしらふへの変化の間で、対象自身には何も変化が起きない。しらふになった瞬間に、砂糖が水に溶けるように、対象の形や組成が変化することはない。しらふは、確かに対象に潜在し特定の状況で顕在化する性質だが、適切な状況であってもしらふを普通だと知覚することは可能である。

このような知覚は、傾向的性質よりかはむしろ多義図形を見るときのようなゲシュタルト的な知覚と同じだと考えられる。ゲシュタルトとは、「部分の加算的集合に還元不可能な全体」(源河(2017):3)と言われる。例えば多義図形を見る際の知覚は、なぜそう見えるのかが多義図形の部分に還元できず部分を全体にどう役割づけるかが異なることによるゲシュタルト的な知覚だと言える。例えばアヒル/ウサギ図形のように、対象のもつ性質が変化していないのにも関わらず、あるときはアヒルに見え、またある時はウサギに見えるというのが多義図形の知覚の特徴である。しらふは、酒していないという身体状態は変化しないのにも関わらず、あるときはしらふと知覚され、別のときは普通だと知覚される。つまり、しらふだと知覚することを、対象の部分に還元して説明することはできない。よって、しらふには、前述の①の特徴に加えて、それが②ゲシュタルト的な知覚によるという新たな特徴があることがわかる。

\subsection{1ー3.しらふは判断や解釈か?}

ではそもそもしらふは性質なのだろうか。しらふをゲシュタルト的な知覚だとすれば、そこで変化しているものは性質の知覚ではなく、判断や解釈にすぎないかもしれない。つまり、感覚的にしらふだと知覚する必要はなく、推論や知識として対象がしらふであることを知っているだけでよいのではないか。またしらふが知覚される性質だとしても、しらふという独立した性質があるのではなく、他の性質の集合を知覚するだけで十分かもしれない。

マクファーソンは、多義図形の知覚が判断や解釈と同一視できない理由を2点に分け説明している。1つ目は、判断や解釈の際に起こる変化の全てが多義図形の知覚における変化につながっているわけではないことである。例えば、ミュラーリヤー図形の知覚の際、たとえ主体がその2つの線の長さを測りそれらが同じ長さであるとわかったとしても、依然としてその長さは異なって見える。よって、判断とは異なる知覚が働いていると考える必要がある。2つ目の理由は、ゲシュタルト的な知覚による知覚の転換は、しばしば自律的に行われるというものである。(MacPherson 2006:91-93)\footnote{マクファーソンの議論をまとめたものとして、源河(2017)も参照した。 (源河2017:192-194)}

しらふの場合にも、これら2つの理由があてはまる。たとえ対象が飲酒していると理解した後でも、対象がしらふであると間違えるときがある。つまり、しらふであるかどうかの判断や解釈とは別にしらふの知覚が行われている。

しかし次に問題になるのは、しらふが判断や解釈とは異なるとしても、それはしらふという性質を知覚していると言えるのかという問題である。しらふが特定の性質の集まりに還元されるのであれば、それを独立した性質として考える必要はない。

結論から言えば、しらふはしらふ以外の性質の組み合わせに還元できないと考えられる。飲酒によって変化する顔色や振る舞いなどは、お酒を飲むとそういう変化を起こす人が多いという傾向に過ぎない。また、顔色や振る舞いなどが全く同じ対象があったとしても、あるときはそれをしらふだと知覚し、別のときはそう知覚しない場合がある。

これらから、しらふは③判断や解釈とも、しらふ以外の性質の集合とも異なる独立した性質だということができる。

\section{第2節 しらふは美的性質か?}
前節では、しらふは、お酒を飲んでいないという身体状態以外にも、①その対象がお酒に関わる環境に属していること及びそれを知覚する側が知っていなければならないこと、②ゲシュタルト的な知覚によって把握されること、③判断や解釈、しらふ以外の性質の集合とも異なる独立した性質であることという3点の新たな特徴を説明しなければならないことを確認した。本節では、これら3つの特徴は美的性質がもつ特徴であり、しらふは美的性質と言えることを示す。これはシブリーの美的概念に関する論及びそこから展開された理論を用い行われる。

シブリーは、本稿が美的性質とよぶような美的用語\footnote{シブリ―がなぜ性質という言葉を用いないかについては、シブリ―の目的が、美的性質を対象がもっているか否かではなく、美的性質を知覚できるか否かにあるからだと考えられる。Sibely(1968)を参照のこと。なおシブリ―のこの論点は、源河(2017)における美的性質の反実在論にも関わる。源河(2017)は、美的性質を「対象と性質の与えられ方」に対応する経験の現象的性質として、美的性質の反実在論の立場をとっている。美的性質の反実在論では、美的性質が実在するかいなかという問題と美的性質が知覚できるかという問題は分けて考えられる。というのは、美的性質を美的経験の現象的性質と考えた場合、対象に内在する形での美的性質は存在しないが、主体がどのように対象を経験するかという仕方で美的性質を知覚することはできるからである。(源河2017:chap.6)本稿も源河の美的性質の反実在論にのっとっている。}に関し、いくつかの特徴づけを行った。シブリーが提示した特徴は、簡単に以下のようにまとめることができる。

\begin{enumerate}
 \item 美的性質の知覚には趣味や鑑賞眼などの特別な能力が求められる。
 \item 美的性質は、非美的性質に依存する。
 \item 美的性質を適用するための十分条件となるような非美的性質はない。
 \item 美的性質の判断のためには、美的性質そのものを知覚することが必要である。
\end{enumerate}

第1部で示したしらふに付与される3つの特徴は、①は(1)に、②は(2)と(3)に、③は(3)と(4)におおよそ対応している。以下ではこれらの美的性質の特徴を詳しく説明し、しらふが美的性質といえるかどうかについて検証していく。

\subsection{2ー1.①お酒に関わる環境についての社会的文化的知識が必要である}

しらふという性質は、それを知覚する人がお酒やお酒を飲む文化などについての知識をもたなければ、しらふではなく普通だとみなされることを1ー1で確認した。このような学習の必要性は美的性質以外の性質でも指摘されるが、しらふの場合の知識はある対象についてだけではなく、対象が取り囲まれている環境についての知識が必要とされる点が異なる。

賞における環境についての知識の必要性は、自然環境の美的鑑賞を論じる分野である環境美学における認知モデルという立場と類似している。環境美学における認知モデルとは、「自然の美的鑑賞には、自然に関する常識的/科学的知識が必要である」とする立場であり、アレン・カールソンが代表的な論者である。

カールソンは、シブリーが提示した美的性質を知覚するための趣味や感受性をさらに発展させ\footnote{青田(2020)も指摘しているように、シブリ―の趣味の必要性から知識を最重視するカールソンの展開は、シブリ―の議論から直接引き出されるものではない。(青田2020:108)しかし本稿では、前述のとおりしらふを「価値記述的な用語」として論じており、その場合対象が何であるかは美的性質の価値に影響すると考えるため、カールソンのと同様、知識をシブリ―の趣味概念の中でも重視する立場をとる。}、自然の美的鑑賞には、「知覚的で情緒的な感受性」と「環境についての一定程度の知識と理解」が必要だとする。(Carlson1976:153)この知識は、「対象が何であるのか」を分類するのに役立つとされる。\footnote{このような、鑑賞の際に対象のカテゴリーを重要視する立場は、シブリーの非美的性質と美的性質の関係性の論点に加え、ウォルトン(1970)が提示した鑑賞の際のカテゴリーの役割についての論から引き継がれている。ウォルトンは、対象がどのカテゴリーに属するかが変化することによって知覚される美的性質が異なることを論じ、対象を適切なカテゴリーで論じることの必要性を論じている。}
のようなカールソンの立場は、カールソンが「風景画モデル」や「物体モデル」と呼ぶような自然の鑑賞を批判するために提案されている。それらのモデルでは、自然は絵画や彫刻のように、形式的な特徴のみでもって鑑賞される。 カールソンはこのような一面的な自然の美的鑑賞を批判し、自然の鑑賞には、五感を用いた環境の知覚ののちに、「美的な重要性への適切な注目と環境の適切な境界線」(Carlson 1979:273)を与える常識的/科学的知識が必要とされるという。つまり、環境の一部分を切り取って鑑賞するのではなく、事物と事物が互いに関係しあい主体及び対象を取り囲むものとしての環境についての知識を得ることが自然環境の美的鑑賞には必要とされる。

しらふという性質もまた、「物体モデル」のように対象をまるで彫刻のようにとらえ、対象の形式的な特徴のみに依拠することでは知覚できない性質だと考えられる。すでに指摘したように、しらふだと知覚するためには、対象がお酒に属する環境にいること及びお酒に関する文化的社会的な知識が必要とされる。 つまり、しらふの知覚における対象が属する環境についての知識の必要性は、環境美学における環境への一定の知識と同列だと言える。これにより、しらふの知覚は、環境美学での美的性質の知覚の一つとして語ることができると思われる。

\subsection{2-2.②ゲシュタルト的知覚である}

1ー2において、しらふは対象の顔色などの性質が一切変化していないのにも関わらず知覚されることから、②しらふを知覚することはゲシュタルト的な知覚だということを指摘した。シブリーもまた美的概念をゲシュタルト的な知覚と類比的に論じるときがある 。\footnote{cf. Sibley1965:140,142,151, Sibely1968:31.}このゲシュタルト的な知覚は、前述のシブリーによる特徴の(2)と(3)に対応している。

ゲシュタルト的な知覚は、対象のもつ色や形などの性質を特定の仕方でまとめあげるような知覚であった。この際、アヒル/ウサギ図形がアヒルもしくはウサギに見えるためには、まず図形それ自体がアヒルとして知覚しうるくちばしの形や、ウサギとして知覚しうる耳の形などの性質を持っていることが前提にある。そのうえで、それぞれの性質を全体に対しどのように役割づけるかによって、その図形をウサギと知覚するかアヒルと知覚するかが変わってくる。

まず、美的性質も多義図形と同様に形や色といった非美的性質に依存する。(Sibely1959:423ー424,Sibely1965,137-138,Sibely1968:35)例えば美的性質を説明するときに非美的性質も指摘したり、絵画のある部分が変化することで、得られる美的性質が異なったりする場合などである。しかしその依存関係から、○○だから優美であるという形で優美さについての一般的な十分条件を取り出すことはできない。このように、美的性質の法則や因果関係を特定できないというのが、美的性質の「非条件支配的」という特徴である。(Sibely1959:424-427)

しらふという性質もまた、もしその対象の一部分が変化してしまったら知覚できなくなる性質である一方で、顔色が赤くないからしらふだというように常に適用しうる条件が見つけられるわけではない。よって、しらふのもつゲシュタルト的な知覚は、美的性質のもつ非美的性質との関係性を満たしていると言える。

\subsection{2-3.③判断や解釈とは異なる独立した性質である}

2ー2で示したシブリーのいう美的性質の非条件支配的という特徴は、4つ目の美的性質の特徴である美的性質という独立した性質を知覚する必要性も説明する。

美的性質が非条件支配的にならざるを得ない理由の1つには、美的性質が常に特定の事例に対応するからというのがある。例えば《ゲルニカ》のもつダイナミックさと、《神奈川沖浪裏》のもつダイナミックさは、同じダイナミックさと言い表せるが、それぞれの美的性質が依拠する非美的性質は全く異なる。

シブリーは、このような美的性質の特性を踏まえ、鑑賞者が美的性質を判断する際には、学習のみによって判断や解釈を知るだけでは不足であり、美的性質を直接知覚する必要があるとしている。(ibid.:432)シブリーは、他者に自身が知覚している美的性質を知覚させようとする営みを「知覚的証明」(Sibely 1965:143)と呼び、美的判断を正当化する方法としてそれらを行う批評家の方法に言及している。(Sibely1959:442-445;Sibely 1968:142-143)

しらふも、この美的性質の特徴によって説明できる。1ー3で見たように、しらふは判断や解釈とは異なる次元で知覚されるのに加え、しらふとは異なる性質に還元することもできない。典型的な特徴はあれど、常に対象ごとに異なる非美的性質に依拠した知覚と判断が行われる。ある対象がしらふであることは、その人がしらふであると他者から聞くことのみによっては保証されず、その対象と主体が接し、直接対象がしらふであることを知覚する必要がある。このような類似から、しらふは前述の美的性質の(3)(4)の特徴も満たしていることがわかる。

\section{結論}

これらの考察から、しらふは美的性質と言えるのではないかと結論する。本稿ではまず第1節でしらふについての分析を通じ、しらふには飲酒していないという身体状態におさまらない特徴を持っていることを説明したのち、第2節で美的性質の特徴について論を進めそのしらふとの類似を確認した。

芸術作品や自然風景に向けられる美的性質とは違い、人間に対する美的性質は倫理と深く関係する。例えばポルノグラフィティやルッキズムなどは美的性質と倫理が最も強く影響し合うものの一つだろう。その時にまず問題となるのは、社会において何が美しいとされているのかという美的性質の内容についての問いである。しかし、その際の分析には、美的性質を対象に付与する行為の倫理性をもまた考える必要があるのではないだろうか。

らふを美的性質とみなすことは、そうした人間に対する美的性質の判断の際に、主体の態度それ自体に目を向ける好例になると思われる。というのは、しらふは「美しい」などよりも非美的性質との結びつきが強く、美的性質の内容の是非が問題になることは少ないからである。このような記述的な美的性質を人間間の交流に見出すことで、美的性質を対象に付与する行為の倫理性を問うことができる。

また、本稿では一貫して美的性質を知覚されるものとして論じてきた。このような知覚段階での倫理性への反省は、感性の学ともいわれる美学独自の視点ともいえるのではないだろうか。
\theendnotes

\begin{thebibliography}{99}
\bibitem{OP}Carlson, A. (1976)”On the Possibility of Quantifying Scenic Beauty”, Landscape Planning 4, pp.131-72.
\bibitem{AN}---(1979) ”Appreciation and the Natural Environment”, The Journal of Aesthetics and Art Criticism , Vol. 37, No. 3, pp. 267-275.
\bibitem{FC}MacPherson, F (2006)”Figures and the Content of Experience”, Noûs, Vol.40, No.1,pp.82-117.
\bibitem{AC}Sibely, F. (1959)”Aesthetics Concepts”, Philosophical Review,Vol.68, No.4,pp.421-50.(フランク・シブリ―(吉成優訳)「美的概念」、西村清和編『分析美学基本論文集』、勁草書房、2015年、99~137頁。)
\bibitem{AN}---(1965)”Aesthetic and Nonaesthetic”, Philosophical Review, Vol. 74, No. 2, pp.135-159.
\bibitem{OA}---(1968)”Objectivity and Aesthetics”, Proceedings of the Aristotelian Society, Supplementary Volumes, Vol.42, pp.31-54.
\bibitem{PA}---(1974)”Particularity, Art and Evaluation”, Proceedings of the Aristotelian Society, Supplementary Volumes, Vol.48,pp.1-21.
\bibitem{CA}Walton, K. (1970)”Categories of Art”, Philosophical Review, Vol.79, No.3, 334-367.(ケンダル・ウォルトン(森功次訳)「芸術のカテゴリー」、電子出版物、2015年、https://note.com/morinorihide/n/ned715fd23434、2020年9月10日閲覧。)
\bibitem{KH}青田麻未『環境を批評する 英米系環境美学の展開』、春風社、2020年。
\bibitem{GK}柏端達也『現代形而上学入門』、勁草書房、2017年。
\bibitem{TH}源河亨『知覚と判断の境界線 「知覚の哲学」基本と応用』、慶應義塾大学出版会、2017年。
\bibitem{PN}西村清和『プラスチックの木でなにが悪いのか 環境美学入門』、勁草書房、2011年。 
\end{thebibliography}

 
\end{document}
