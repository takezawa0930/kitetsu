%\documentclass[10pt,a4j]{utjarticle}
\documentclass[b5j,twoside,twocolumn]{utarticle}
%\documentclass[b5j,twoside]{utarticle}
%\documentclass[b5j,twoside,twocolumn]{utbook}
\setlength{\columnsep}{2zw}
\usepackage{bxpapersize}
\usepackage{pxrubrica}
\rubysetup{<hj>}
\usepackage{endnotes}
\usepackage{multicol}
\usepackage{plext}
\renewcommand{\theendnote}{[後注\arabic{endnote}]}
\renewcommand{\thefootnote}{\arabic{footnote}}
\usepackage{pxftnright}
\usepackage{fancyhdr}
\setlength{\topmargin}{5mm} % ページ上部余白の設定(182mm x 257mmから計算)。
\addtolength{\topmargin}{-1in} % 初期設定の1インチ分を引いておく。
\setlength{\oddsidemargin}{21mm} % 同、奇数ページ左。
\addtolength{\oddsidemargin}{-1in}
\setlength{\evensidemargin}{17mm} % 同、偶数ページ左。
\addtolength{\evensidemargin}{-1in}
\setlength{\footskip}{-5mm}
%\setlength{\marginparwidth}{23mm}
%\setlength{\marginparsep}{5mm}
\setlength{\textwidth}{225mm} % 文書領域の幅(上下)。縦書と横書でパラメータ(width / height)の向きが変わる。
%\setlength{\textheight}{150mm} % 文書領域の幅(左右)
\makeatletter
\def\@cite#1#2{\rensuji{[{#1\if@tempswa , #2\fi}]}}%%
\def\@biblabel#1{\rensuji{[#1]}}%%%
\makeatother
\usepackage{enumerate}
\usepackage{braket}
\usepackage{url}
\usepackage[dvipdfmx]{graphicx}
\usepackage{float}
\usepackage{amsmath,amssymb}
\newcommand{\relmiddle}[1]{\mathrel{}\middle#1\mathrel{}}
\usepackage{ascmac}
\usepackage{okumacro}
\usepackage{marginnote}
%\usepackage[top=15truemm,bottom=15truemm,left=20truemm,right=20truemm]{geometry}
\usepackage{cleveref}
\usepackage{plext}
\usepackage{pxrubrica}
\usepackage{amsmath}
\usepackage{fancybox}
\usepackage[dvipdfmx]{graphicx}
\usepackage{cancel}
\setcounter{tocdepth}{3}

%\renewcommand{\labelenumi}{(\Alph{enumi})}
\usepackage {scalefnt}
\makeatletter
\@definecounter{yakuchu}
\@addtoreset{yakuchu}{document}% <--- depende on class file
\def\yakuchu{%
\@ifnextchar[\@xfootnote %]
{\stepcounter{yakuchu}%
\protected@xdef\@thefnmark{\theyakuchu}%
\@footnotemark\@footnotetext}}
\def\yakuchutext{%
\@ifnextchar [\@xfootnotenext %]
{\protected@xdef\@thefnmark{\theyakuchu}%
\@footnotetext}}
\def\yakuchumark{%
\@ifnextchar[\@xfootnotemark %]
{\stepcounter{yakuchu}%
\protected@xdef\@thefnmark{\theyakuchu}%
\@footnotemark}}
\makeatother

\usepackage{atbegshi,etoolbox}

\newcounter{newfoot}
\patchcmd{\footnotetext}{\thempfn}{\thenewfoot}{}{}

\newcommand{\evenfootnote}[1]{%
  \ifodd\value{page}%
    \footnotemark%
    \AtBeginShipoutNext{%
      \stepcounter{newfoot}\footnotetext{#1}%
    }%
  \else%
    \stepcounter{newfoot}\footnote{#1}%
  \fi%
}


\pagestyle{fancy}

\title{哲学的対話を発生させるロボットシステム------社会インフラへ向けた予備的考察}
\author{五十里翔吾}
\date{\vspace{-5mm}}
\setcounter{page}{101}

\begin{document}
\maketitle

\setlength{\footskip}{-2mm}
\lhead[]{【エッセイ】}
\chead[]{}
\rhead[哲学的対話を発生させるロボットシステム------社会インフラへ向けた予備的考察]{}
\lfoot[]{\thepage{}}
\cfoot[]{}
\rfoot[\thepage{}]{}

\let\yakuchu=\endnote
\renewcommand{\footnoterule}{\noindent\rule{100mm}{0.3mm}\vskip2mm}
%\tableofcontents
\thispagestyle{fancy}
\section{はじめに}
本稿の目的は、日常の中に哲学的対話を生じさせるシステムを社会実装するための構想を提出することである。さまざまな実践\footnote{引用}が示している通り、日常における哲学的対話は、生活にポジティブな影響をもたらす。それは、自らの価値観を見直すきっかけとなったり、相互行為における他者への尊重を促したりする。


哲学的対話は、以下の点でその他の言語コミュニケーションと区別される。
\begin{enumerate}
\renewcommand{\labelenumi}{\pbox<y>{(\arabic{enumi})}}
\item 参与者間で一つの問いを共有して、それを追求する。
\item 「話に途中で割り込まない」「話すときは手を挙げる」などのルールがある。
\end{enumerate}
これらは、哲学的対話の場を日常から区別し、より自由な発言を促すために設けられている。


これまで、哲学カフェや哲学カウンセリングなど、さまざまな形で哲学的対話の実践が行われてきた。現時点では、ある日時・場所に、専門家である哲学者が主導する形でイベントが開催されるという運営の形態が主であり、万人にとってアクセスしやすいものではない。しかし、哲学的対話は、社会における十全な相互行為の基盤となる価値を提供するものである。よって、対話の機会は、社会のインフラとして整備される必要があると考えられる。


本稿は以上のような関心をもって書かれている。


2章では、グレゴリー・ベイトソンの理論に基づいて、哲学的対話によって自己変容がもたらされる過程を記述する。
3章では、2章の分析を踏まえて、哲学的対話をインフラとして整備するために必要な仕様を提案する。



\section{哲学的対話のシステム論}
哲学的対話を発生させるシステムを開発するためには、十全な哲学的対話においては何が生起するのかを明らかにする必要がある。
本章の目的は、哲学的対話がいかにして人々に影響を与えるのかを記述することである。哲学的対話を、人々の生活の中で持続する自己変容のコミュニケーションと捉え、それが成立している状況はどのように描写できるのかをシステム論の立場から検討する。


\subsection{自己変容のコミュニケーション理論}
哲学的対話とはどのような相互行為だろうか。それはまず、対話の一種である。それでは、「対話」という語はどのように説明されているのだろうか。


劇作家の平田オリザは、対話と会話を区別して以下のように整理している\footnote{平田オリザ、『対話のレッスン 日本人のためのコミュニケーション術』、講談社、二〇一五}。
\begin{description}
\item{「会話」} はお互いの細かい事情や来歴を知った者同士のさらなる合意形成に重きを置く、すでに知り合った者同士の楽しいお喋りのことである。
\item{「対話」} は異なる価値観のすり合わせであり、差異から出発したコミュニケーションの往復に重点を置く。
\end{description}
この説明は、用法の説明というよりはむしろ術語の定義である。術語は、特定の前提のもとで主張や議論を行うために概念を整理するものだ。人の相互行為を說明しようと試みると、このようにある種遂行的な描写を行うことになるだろう。
この分類に異論はない。しかし、本章の目的においては、まだ不十分である。なぜなら、本章で検討したいのは、言語を介したある種の相互行為が、参与者に対してもたらす持続的な影響だからだ。


マルティン・ブーバーの思想\footnote{彼の思想は以下の著書に収められている。マルティン・ブーバー、植田重雄訳、『我と汝・対話』、岩波書店、一九七九}は、「対話の思想」と言われる。ブーバーの関心は「対話が人に及ぼす影響」であった。ブーバーは、人が世界を〈われ―それ〉〈われ―なんじ〉という二通りの方法で「語る」と主張し、二者が「互いに向かい合うこと」を根底において、人のあるべき生を論じた。本章の目的はブーバーの関心と近いが、ブーバーの神学的な議論からは本稿の最終目的である「システムの実装」に関する示唆を引き出すことは容易ではない。本章では、同様の着想から、サイバネティックス理論を応用してシステム論的なコミュニケーション理論を提示したベイトソンの議論に従って哲学的対話を定義したい。


\subsubsection*{ベイトソンのコミュニケーション理論}
ベイトソンのコミュニケーション理論の核をなすのは、サイバネティックス認識論と論理階型論である。以下、簡単に紹介する。\footnote{文献\cite{JIKO}を参考に、ベイトソンの著書\cite{SEM}を整理した。} \\
\textbf{(1)サイバネティックス認識論}


この認識論によると、一人の生きた人間の、生きた現実にとって「存在すること」と「認識すること」は切り離すことができず、それらはまさに「行為すること」である\footnote{世界とはこういうものだ(what sort of world it is)という(通常無意識レヴェルの)思い込みが、世界をどのように捉えそのなかでどうふるまうか(how to see and act)ということを決定するわけだし、逆に、かれの知覚と行動のあり方(how)が、世界の何であるか(what)を決定するわけである。}。行為とは、それ自身が自己を修正する自己言及的な過程である。その過程は円環的な回路として捉えられ、その上を流れる差異は別の差異と交わり、再帰的に参照される。ベイトソンは、この過程を\textbf{精神(mind)}という。\\
\textbf{(2)論理階型論}


論理階型論は、認知が階層的であるということを主張する。ベイトソンによると、世界の認知は四段階の学習によって成立している。ゼロ学習は、直接的な刺激の学習である(パブロフの犬)。学習Ⅰは、状況に依存したシグナルを学習する過程である。この学習により、メッセージをコンテクストに照らし合わせて解釈できるようになる。学習Ⅱは、さらなる自己の統合の過程である。「個々の状況を表すパターン」がどのような法則によって生じているのかを学習する(=世界の構造的認知過程、すなわち通常自己と言われるもの)。ベイトソンは、さらに学習Ⅲを定義した。このレベルでは、宗教的「回心」のように、自己そのものが組み替えられる。


世界に働きかけるシステムは、このような階層的な学習を行う。ゼロ学習と学習Ⅰは、我々の個別の\textbf{行為}に関わる。学習Ⅱは、我々が\textbf{自己}と呼ぶ、「その人に染み込んださまざまの前提(S.E.M p.432)」を構成する。そして、学習Ⅲは、「自己」と世界の\textbf{全体}の関わりを規定する。システム論を用いると、この関係は以下のように整理することができる\cite{JIKO}。

(Ⅰ)行為システムは(Ⅱ)自己システムのサブシステムである。そして、自己システムをその上位で(Ⅲ)エコシステムが包摂している。一般システム論によると、部分による部分の修正は不可能である。自己システムの変容は、(a)自己システムの一部がエコシステムからのフィードバックにより変容を受ける、あるいは(b)自己システムにおけるパターンに亀裂が走り、それによってより大きな回路(Ⅲ)における学習Ⅲが生じる(自己の組み換え)という形をとることになる。


これらの学習の過程を理解するには、ベイトソンによる「関係の区別」を導入する必要がある。
ベイトソンは、物事の間の関係を「対称的」なものと「相補的」なものに二分した。
\begin{quote}
二者関係において、AとBの行動が、(AとBによって)同じものとして見られ、しかもAの行動の強まりがBを刺激して同じ(とされる)行動を強め、逆にまたBの行動がAの〝同じ〟行動を促進するようなかたちで二つが連関しているとき、それらの行動に関して両者の関係は「対称的」(symmetric)であるという。


一方、たとえば見る行為と見せる行為とが互いにフィットするように、AとBの行動が同じではないが相互にフィットするものであり、しかもAの行動の強まりがBの行動の強まりを呼ぶようなかたちで両者が連関しているとき、それらの行動に関して両者の関係は「相補的」(complementary)であるという。%p462
\end{quote}


ベイトソンは、(a)の例として芸術鑑賞の経験を挙げている。我々の世界の認識は、無意識下に沈んだ認知枠組みに(例えば遠近法)よって支えられている。芸術作品においては、その無意識のゲシュタルトが主題化されることがある(例えばキュビズム絵画)。そのような作品が伝える無意識のゲシュタルトについての情報は、(Ⅲ)エコシステムの回路に対してフィードバックされる。このようにして、世界の見方が「全体論」的なものに変化し、自己のより高い統合が促される\footnote{ベイトソンは、芸術のこのような作用を「精神を癒やすものとしての芸術」(corrective nature of art)と呼んだ。}。


(b)の例として挙げられているのは、アルコール依存症患者が耽溺から解放されるときの体験である。AA〈アルコホリックス・アノニマス〉におけるアルコールとの戦いは、「十二のステップ」からなる。その最初の二つは以下のようなものである。\footnote{AA日本ゼネラルサービス\rensuji{HP}より。\url{https://aajapan.org/12steps/}}
\begin{enumerate}
\item 私たちはアルコールに対し無力であり、思い通りに生きていけなくなっていたことを認めた。
\item 自分を超えた大きな力が、私たちを健康な心に戻してくれると信じるようになった。
\end{enumerate}

アルコール依存症患者は、(時に周囲から言われ)誘惑と戦うべきだと信じている。そうして生じる「自らの意思でアルコールに打ち勝つことを示してやる」という酒との「対称的」関係が自己を破滅に向かわせているのだと自覚することが、解放へのはじめの一歩なのだ。この自体は、以下のように説明できる。(Ⅱ)自己システムが「対称的」な関係のもとでは維持できないことが分かったとき、すなわち自己がある種の「絶望」に追いやられたとき、より上位のシステム(Ⅲ)における自己の組み換えが発生する。そして、患者は自己を酒との「相補」的な関係の中に位置づけられるようになる。アルコール依存症患者の「開放=回心\footnote{ベイトソンは、患者の経験する解放を一種の神学的体験と考えていた。}」は、このように説明される。


\subsubsection*{「哲学的対話」に向けて}
ベイトソンのいう(a)、(b)の学習による自己変容は、ある種の相互行為を通した自己変容を記述している。これらのどちらの経験も、(Ⅱ)自己システムに学習されたパターンが維持されなくなることで、自己を包摂するより大きなシステムとの「相補的」関係が生じるという事態がその契機であった。
それでは、これらの枠組みは哲学的対話の満足な定義を与えるだろうか。もう少し距離があるように考えられる。


(ⅰ)哲学的対話を駆動しているのは、何よりもまず、問いとの「対称的」関係である\footnote{いわゆる「哲学的混乱」という状態である。}。「相補的」態度\bou{しか}存在しないのであれば、そもそも探究は行われない。一方で、複数人での議論においては、一つのもっともらしい考えが提出されると、皆がそれを受け入れるようになることがある\footnote{社会心理学の理論である「集団極性化」と「フリーライディング」によって説明できる。}。このような現象が生じる背景には、「対称的」なテーマが入り込んでいる。すなわち、「どちらが正しいのだろう」という比較を行う相互行為が前提とされている。ある考えが別の考えに「論破」される場合にそれは顕著である。また、もっともらしい意見を聞いて黙ってしまう、あるいは考えるのをやめるという場合にも、潜在的には、複数の考えを比較してより正しい方を選ぶというディベートの構造がある。


すなわち、「相補的」関係のみでは哲学的対話は成立できず、「対称的」関係が支配する場では継続が難しくなる。このことが示すのは、哲学的対話においては、「対称的」関係と「相補的」関係が同時に成立しているということだ。


(ⅱ)哲学的対話によって生じる自己の変容は、おそらく不可逆的で断絶を伴うものではない。自己に弾力を与え、不確実な状態に対する耐性を高めさせるものである。そこで生じるのは、他者との言語を介した相互行為を経て得られた自らの「認識=存在論」についての問いを、他者の声をフィードバックさせながら考え追求する、という過程である。%この過程は、どのようにして自己システムの変容をもたらすのだろうか。
すなわち、哲学的対話がもたらす自己の変容は、回路に情報が流れることそれ自体(学習Ⅲ)によってではなく、回路間の連絡が持続することによってもたらされる、ということである。


以上を踏まえると、哲学的対話の中で生じている相互行為と、その相互行為が人の自己変容を促す過程は以下のように記述できる。


\begin{enumerate}
\renewcommand{\labelenumi}{\pbox<y>{(\arabic{enumi})}}
\item 哲学的対話において、参与者は①他者(他の参与者)との「相補的」関係+世界との「対称的」関係、あるいは②他者との「対称的」関係+世界との「相補的」関係、のどちらかの状態にあり、それらの間を動的に遷移する。このとき、
「他者」あるいは「世界」のうち(Ⅱ)自己システムと「相補的」関係にあるものが、(Ⅲ)エコシステムから流れ出し、自己システムへと連絡する。しかし、別の一方とは「対称的」関係を結んでおり、その関係が挫折するわけではないので、学習Ⅲによる自己の組み換えは行われない。
\item 哲学的対話の中で共有された主題と、「他者」「世界」と自己システムとの、\bou{部分的に「対称的」で部分的に「相補的」}な関係は、
そこで生じた自己システムとエコシステムとの連絡と共に、全体システムの回路に「焼き付け」られる。
\item そして、「対称的」な関係を含む「焼き付け」によって、対話の現場での相互行為が終了した後も、それぞれの参与者が日常の中で主題についての追求を行うことができるようになる。それが新たな哲学的対話\pbox<y>{(1)}を誘発する。
\item 同時に、自己システムとエコシステムの連絡が保たれることで、
エコシステムにおいて無意識下に沈んでいた、他者との相互行為において前提とされている認知枠組みや、世界についての無意識のゲシュタルトを意識的に思考の対象にできるようになる。これによって、部分的、表面的ではなく全体論的な世界の見方が獲得され、より統合された自己システムが達成される。
\end{enumerate}


\subsection{自己変容のシステム}
前節では、哲学的対話の中で生じている相互行為を記述することを試みた。そこで言われている
①他者(他の参与者)との「相補的」関係+世界との「対称的」関係、②他者との「対称的」関係+世界との「相補的」関係、とはどのような状態で、それらの状態の遷移はどのように生じるのだろうか。


まず、①が表す状況は以下のようなものである。参与者は、自身が認識する世界についてのある問いに対して、少数の理解可能な「原理」によって答えが与えられると想定する。そして、他者と異なる視点や前提に立った、多様な考えを提出するように努める。


一方、②が表す状況は以下のようなものである。参与者は、「今までに出た考えから統一的な答えを得ることは不可能だ」という考えを持ちつつも、他者の考えと自分の考えを比較し、より良い答えを提出しようと自分自身で努める。


個人の中での、これらの状態間の遷移は、哲学的対話というより大きなシステムの挙動によって決定される。すなわち、多様な考えが提出されている状態では、②が目指され、意見が少数に統合されようとしている局面では①が目指される。
これらのどちらの状況でも、自己システムの内部はダブル・バインドの状態にある。しかし、哲学的対話というシステムはその矛盾によって維持されるのである。
そして、哲学的対話システムによって形成される「自己変容の回路」は、一度それが形成されると、他者との現場での相互行為が終了しても、自己システムの上位に保存され、自己システムの統合を促進する。


「自己変容の回路」が形成され、維持されるためには現場での十全な実践が行なわれ、それが繰り返されることが不可欠である。
%(ⅰ)①、②を満たすな対話が現場で行われ、(ⅱ)その後の生活の中でも活性化される必要がある。すなわち、上記の哲学的対話の回路が起動する必要がある。
%(ⅰ)に関しては、哲学カフェなどの実践の場が該当する。(ⅱ)が指すのは、日常生活の中での雑談において哲学的な問いが検討される、というような状況である。
しかし、実践の場に参加することには難しさが伴うし、そのような場が十分にあるわけではない。さらに、日常生活で哲学的な雑談が行なわれる機会は、そう多いとは考えにくい。次章では、この問題を解決するため、哲学的対話を発生させる、ロボットを用いたインフラとしての対話システムを提案する。

\section{哲学的対話ロボットシステム}
%前章では、哲学的対話が生じている状態では、どのような相互行為が行われ、参与者のその後における持続的な影響をもたらすかのを記述した。
本章の目的は、哲学的対話を生じさせることを目的としたロボット対話システムに必要な仕様を提出することである。





前章で明らかになったことは、哲学的対話は生活の中で繰り返し行なわれる必要があること、そして哲学的対話においては、「対称性」と「相補性」が混在する二つの状態を遷移することが重要であることである。


ここで提出するシステムは、これらの要件を満たすものである。
全体システムは三つの部分からなる。
\subsubsection*{(ⅰ)街中で人に問いかけて考えを引きだす「自律ロボット」}
このシステムは、前章の①の対話状況を目指すものである。ロボットは駅や学校、遊園地などを徘徊し、道行く人に哲学的な問いかけをする。ここでの問いは、後述する(ⅲ)の対話の木から抽出され、人が答えた内容は、そのスタンス(問いに対する答えなのか、何らかの意見に対する賛成あるいは反対なのか、別の形式の意見なのか)と共に、対話の木に登録される。


人は問われることで答えを求めるようになる(問いとの「対称的」関係)と考えられる。また、日常の中で問われることで、その人の生活に根ざした多様な考えが引き出せる(他者との「相補的」関係)。


このシステムにおいてロボットを対話のインターフェースに用いることには、以下のようなメリットがある。
\begin{enumerate}
\renewcommand{\labelenumi}{\pbox<y>{(\arabic{enumi})}}
\item 発話速度や音量、また使用する言語やコミュニケーションの形態そのものを、人の労力を要することなく調整でき、より多くの人とのコミュニケーションが可能になる。
\item ロボットを相手にすると人は自己開示を行いやすくなる場合があることが知られており\cite{SHIMA}\cite{KAIJI}、より深い意見を引き出せる可能性がある。
\end{enumerate}

\begin{figure}[h]
\centering
\begin{tabular}<y>{c}
\begin{minipage}[c]{0.65\hsize}
\centering
\includegraphics[scale=0.5]{system1}
\caption{ソクラテスとプラトン}
\end{minipage}
\end{tabular}
\end{figure}

ところで、二体のロボットを用いたほうが、ロボットと人のコミュニケーションが自然に行えるという研究がある\cite{MUL}。ゆえに、このシステムは、複数のロボットを連携させて人に問いかけを行うことで、より意見を引き出しやすくなると考えられる。(二体のロボットの名前は「ソクラテス」と「プラトン」としてはどうだろうか。)

\subsubsection*{(ⅱ)ロボットに仲介された「タッチパネル越しの対話システム」}
このシステムは、前章の②の対話状況を目指すものである。


このシステムは、三人程度で使用するシステムである。駅や公園などの公共の場に設置されることを想定している。
対話の場には、ロボットが参与する人の数だけ設置されている。人はタッチパネルを持ち、自分のロボットに発話させる内容を表示された選択肢の中から選ぶ。ここで表示される選択肢は、対話のシナリオに沿ったものになる。
対話のシナリオは、対話の場にいる人が過去に(ⅰ)のシステムを通して登録した意見に言及しながら(ⅲ)対話の木を展開していくという形式をとる。すなわち、対話の参与者は、自分がかつて発した考えや、その発話に関連する主張を、自分あるいは他の参与者が操作するロボットの発話を通して聞くことになる。
この対話の場では、他者が普段の生活の中で考えた多様な意見に触れることになる。よって、それらを比較して深めるという形式(他者との「対称的」関係)の対話が促され、対話の中で局所的な結論が目指されることが少なくなる(問いとの「相補的」関係)と考えられる。


タッチパネルのインターフェースは、以下のような選択肢を表示する。




対話が進むと、

%このしすてむはなんでいいのか


対話の始まりが選択式のシナリオに基づいていることで、「考えながら話す」あるいは「話しながら考える」ことの難しさが緩和される。また、表示される選択肢には自分がかつて考えたことが含まれることになる。ゆえに、このシステムを使用すると、「考える」プロセスと「話す」プロセスが分離された状態から対話が始まる。


この対話システムは、

%%%%%%%そして、このようにして対話が展開されると、そこで出た考えを踏まえてそれらを統合する考えや、新たな意見が形成される可能性がある。タッチパネルインターフェースには、自ら意見を語るという選択肢も表示される。選択肢にない考えを言いたい場合にそれを選択し、自ら語ることができる。次第に、参与者はロボットを通さずに直接対話を行うようになるだろう。


%多様な意見に触れた状態で対話を行うため、

以上のようなステップを経ることで、ある程度前提が共有された状態で、言語コミュニケーションが行える。これにより、対話に齟齬が生じるリスクが下がり、より深い議論が行いやすくなるとも考えられる。
さらに、このような機能があることで、見ず知らずの人とも哲学的対話が行える可能性がある。ゆえに、このシステムは、ソーシャル・キャピタルの醸成に寄与する可能性もある。


このシステムにおいてロボットをインターフェースに用いると、以下のようなメリットがある。
\begin{enumerate}
\renewcommand{\labelenumi}{\pbox<y>{(\arabic{enumi})}}
\item 上で述べたように、コミュニケーションの方式を多様にすることができる。
\item 選択式の対話に、ロボットのジェスチャーや視線などのモダリティが加わることで、参与感が増す。
\item 自分の意見がロボットを通して語られることで、客観視が可能になる。
\end{enumerate}


\begin{figure}[h]
\centering
\begin{tabular}<y>{c}
\centering
\begin{minipage}[c]{0.65\hsize}
\centering
\includegraphics[scale=0.55]{system2}
\caption{タッチパネル越しの対話システム}
\end{minipage}
\end{tabular}
\end{figure}

この対話の場は、自分が話さなくともロボット同士の言語的なコミュニケーション(にみえるもの)が単純な操作で生起するため、「気まずくない」場となると考えられる。そのような場を先に共有していることで、その場で考えたことを口頭で話す場合に話しやすくなることがあるかもしれない。




\subsubsection*{(ⅲ)問いに対する人々の考えを構造化する「哲学的対話の木」}
哲学的対話の木は、さまざまなトピックについての人々の考えの連関を木構造で整理したものである。
この対話の木を通して、(ⅱ)のシステムを通した対話のファシリテーションが間接的に行なわれる。

まず、大元となる木は、哲学的対話の専門家によって、既存の議論を踏まえて構成される。さらに、専門家の介入によって、(ⅰ)の自律ロボットを通して登録された考えに対してさらに問いかけたり別の観点を提示する発話を登録して、木を発展させる。
上で述べたとおり、(ⅱ)のシステム上で行なわれる対話のシナリオは、このようにして組織化される哲学的対話の木を展開することによって構成される。
ゆえに、個々の現場には専門家が介在しなくとも、間接的なファシリテーションが行なわれることになる。
%このようにして組織化される哲学的対話の木は、それが展開されることで、%(ⅱ)のシステム上で行なわれる対話を方向づける。
%このようにして組織化される哲学的対話の木を展開するという方式で(ⅱ)の対話システムのシナリオが構成されることで、(ⅱ)のシステムを介して
%を構成することで、個々の現場に専門家が居なくても、専門知の恩恵を受けることができるようになる。


上記の(ⅰ)、(ⅱ)のシステムがどのトピックを選択するかは場所や時期によって変化させる。すなわち、今週の大阪は「死」で、来週の広島では「友達」というトピックが選ばれる、といった具合に。
すなわち、複数の哲学的対話の木が社会の中で自己組織化してゆく。
このようなシステムが存在するということが社会で共有されること自体によって、(ⅰ)、(ⅱ)のインターフェースを介さずとも、日々の会話の中で問いを共有した対話が発生しやすくなるという可能性もある。
%哲学的対話が発生しやすくなるという可能性もある。\\
\begin{figure}[h]
\centering
\begin{tabular}<y>{c}
\begin{minipage}[c]{0.65\hsize}
\centering
\includegraphics[scale=0.5]{system3}
\caption{対話の木}
\end{minipage}
\end{tabular}
\end{figure}

以上の三つのシステムによって構成される哲学的対話ロボットシステムを、哲学的対話の機会を提供するための社会インフラとして提案する。

%もちろん、現在の技術では、特に(ⅰ)など、実装が難しいものもある。本稿では立ち入った検討は行えないが、今後は哲学プラクティス、ロボット工学、自然言語処理などの研究者が連携することで、技術を発展させていく必要があるだろう。
%もちろん、現在の技術では、特に(ⅰ)など、実装が難しいものもある。%小中学校や介護施設などのコミュニティをフィールドに技術開発と実証研究を


\section{おわりに------対話ロボットの未来}
本稿では、哲学的対話が成立する場面で、どのようにして自己変容がもたらされるのかを、ベイトソンのコミュニケーション理論に依拠して記述した。そして、その記述によって明らかになった要求を満たす哲学的対話システムを提案した。このシステムは、インフラとしての社会実装を目指すものである。


この対話システムには、ロボットというインターフェースが用いられる。このあり方は、今後社会に実装される対話ロボットの一つの未来を示しているようにも考えられる。本稿の結びとして、この点について簡単に触れる。人に理解できる言語を発する対話ロボット\footnote{認知科学者の岡田美智男が手掛ける「むー」のように、人と同じ言語を発することをそもそも目指していないものもある。弱いロボット:「自らの弱さや不完結さを適度に開示しつつ、周囲の人からの手助けを上手に引き出しながら、一緒になって合目的的な行為を組織していくロボット」\url{https://www.icd.cs.tut.ac.jp/}}は、人に似せようとして設計されてきた。しかし、人とロボットは生理的特徴を共有していないし、同じ社会的文脈に置かれることもない。そして、電源を切ると無残にもうなだれてしまう。一方で、多くの日常会話は、お互いの立場や職業などに依存するか、生活の中で生じる経験に立脚して発展する。Pepperくんに話しかけたときに感じる虚しさは、このようなズレに起因するのではないだろうか。だとすると、技術が進歩しても、この溝はなかなか埋まるものではないのではないか。


哲学的対話においては、可能な限り、習慣や日常的な拘束力はその場から排除されるように努められる。そこでなら、対話ロボットはまたとないパートナーになれるかもしれない。その「のっぺらぼう」さを利用して、日常のさまざまな「あたりまえ」を疑うように迫るのだ。\\



\section*{参考文献}%追加します。
\begin{thebibliography}{99}
{\small
\bibitem{JIKO}亀山佳明「自己変容のコミュニケーション------G・ベイトソン・ノート--」香川大学一般教育研究、ニ〇一二、\rensuji{33}巻 二三七〜二五七頁
\bibitem{SEM}
\bibitem{KAIJI}\pbox<z>{内田 貴久,高橋 英之,伴 碧,島谷 二郎,吉川 雄一郎,石黒 浩 (2017)}\\\pbox<z>{ロボットによる傾聴を通じた自己開示の促進,\footnotesize 日本認知科学会第34回大会}
\bibitem{SHIMA}\pbox<z>{Kumazaki H, Warren Z, Swanson A, Yoshikawa Y, Matsumoto Y, Tak}\\\pbox<z>{ahashi H, Sarkar N, Ishiguro H, Mimura M, Minabe Y and Kikuchi M}\\\pbox<z>{ (2018) Can Robotic Systems Promote Self-Disclosure in Adolescents}\\\pbox<z>{ with Autism Spectrum Disorder? A Pilot Study. Front. Psychiatry}\\\pbox<z>{ 9:36. doi: 10.3389/fpsyt.2018.00036}
\bibitem{MUL}\pbox<z>{Arimoto, T., Yoshikawa, Y. \& Ishiguro, H. Int J of Soc Robotics}\\\pbox<z>{ (2018) 10: 583. https://doi.org/10.1007/s12369-018-0468-5}
\bibitem{TOU} 小川先生のやつ

}
\end{thebibliography}
%対話ロボットのあり方なのでは?

%哲学はどこからでも始まるので、このシステムを日常会話とどうつなぐか



\end{document}
