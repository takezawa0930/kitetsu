%\documentclass[10pt,a4j]{utjarticle}
\documentclass[b5j,twoside,twocolumn]{utarticle}
%\documentclass[b5j,twoside]{utarticle}
%\documentclass[b5j,twoside,twocolumn]{utbook}
\setlength{\columnsep}{2zw}
\usepackage{bxpapersize}
\usepackage{pxrubrica}
\rubysetup{<hj>}
\usepackage{endnotes}
\usepackage{multicol}
\usepackage{plext}
\renewcommand{\theendnote}{[後注\arabic{endnote}]}
\renewcommand{\thefootnote}{\arabic{footnote}}
\usepackage{pxftnright}
\usepackage{fancyhdr}
\setlength{\topmargin}{5mm} % ページ上部余白の設定(182mm x 257mmから計算)。
\addtolength{\topmargin}{-1in} % 初期設定の1インチ分を引いておく。
\setlength{\oddsidemargin}{21mm} % 同、奇数ページ左。
\addtolength{\oddsidemargin}{-1in}
\setlength{\evensidemargin}{17mm} % 同、偶数ページ左。
\addtolength{\evensidemargin}{-1in}
\setlength{\footskip}{-5mm}
%\setlength{\marginparwidth}{23mm}
%\setlength{\marginparsep}{5mm}
\setlength{\textwidth}{225mm} % 文書領域の幅(上下)。縦書と横書でパラメータ(width / height)の向きが変わる。
%\setlength{\textheight}{150mm} % 文書領域の幅(左右)
\makeatletter
\def\@cite#1#2{\rensuji{[{#1\if@tempswa , #2\fi}]}}%%
\def\@biblabel#1{\rensuji{[#1]}}%%%
\makeatother
\usepackage{enumerate}
\usepackage{braket}
\usepackage{url}
\usepackage[dvipdfmx]{graphicx}
\usepackage{float}
\usepackage{amsmath,amssymb}
\newcommand{\relmiddle}[1]{\mathrel{}\middle#1\mathrel{}}
\usepackage{ascmac}
\usepackage{okumacro}
\usepackage{marginnote}
%\usepackage[top=15truemm,bottom=15truemm,left=20truemm,right=20truemm]{geometry}
\usepackage{cleveref}
\usepackage{plext}
\usepackage{pxrubrica}
\usepackage{amsmath}
\usepackage{fancybox}
\usepackage[dvipdfmx]{graphicx}
\usepackage{cancel}
\setcounter{tocdepth}{3}

%\renewcommand{\labelenumi}{(\Alph{enumi})}
\usepackage {scalefnt}
\makeatletter
\@definecounter{yakuchu}
\@addtoreset{yakuchu}{document}% <--- depende on class file
\def\yakuchu{%
\@ifnextchar[\@xfootnote %]
{\stepcounter{yakuchu}%
\protected@xdef\@thefnmark{\theyakuchu}%
\@footnotemark\@footnotetext}}
\def\yakuchutext{%
\@ifnextchar [\@xfootnotenext %]
{\protected@xdef\@thefnmark{\theyakuchu}%
\@footnotetext}}
\def\yakuchumark{%
\@ifnextchar[\@xfootnotemark %]
{\stepcounter{yakuchu}%
\protected@xdef\@thefnmark{\theyakuchu}%
\@footnotemark}}
\makeatother

\usepackage{atbegshi,etoolbox}

\newcounter{newfoot}
\patchcmd{\footnotetext}{\thempfn}{\thenewfoot}{}{}

\newcommand{\evenfootnote}[1]{%
  \ifodd\value{page}%
    \footnotemark%
    \AtBeginShipoutNext{%
      \stepcounter{newfoot}\footnotetext{#1}%
    }%
  \else%
    \stepcounter{newfoot}\footnote{#1}%
  \fi%
}


\pagestyle{fancy}

\title{「哲学的混乱」を持続させる対話システムへ向けた予備的考察}
\author{五十里翔吾}
\date{\vspace{-5mm}}
\setcounter{page}{101}

\begin{document}
\maketitle

\setlength{\footskip}{-2mm}
\lhead[]{【論考】}
\chead[]{}
\rhead[「哲学的混乱」を持続させる対話システムへ向けた予備的考察]{}
\lfoot[]{\thepage{}}
\cfoot[]{}
\rfoot[\thepage{}]{}

\let\yakuchu=\endnote
\renewcommand{\footnoterule}{\noindent\rule{100mm}{0.3mm}\vskip2mm}
%\tableofcontents
\thispagestyle{fancy}
\section{はじめに}
本稿の目的は、「哲学的混乱」を生じさせ、それを持続させるようなシステムを社会実装するための構想を提出することである。例えば、「箱とは〝真には〟何なのか」という問いは、哲学的混乱への入り口である。哲学的混乱と結びついた思考傾向とは、「一般的なもの」に対する渇望であるとか、異なる主題群を(少数の)何かに帰着させ説明してみせたいという誘惑だと言われる。哲学の目的は、このような混乱を治療することであるとウィトゲンシュタインは主張している。個々のケースでそれぞれの言葉がどのように使われているのかを研究して満足せよと。しかし、彼のいう混乱状態が人のさまざまな探究を駆動してきたということは確かである。そして、探究は自己の統一を可能にする。


\section{哲学的混乱のナラティヴ}
%本章では、哲学的混乱の状態にあることで生じ得るポジティブな効果を説明する。
\section{}

\section{対話と自己変容}
本章の目的は、「哲学的混乱」が持続している状況を描写することである。まず、「哲学的混乱」が持続するためには、集団内での相互行為が重要であることを示す。そして、「哲学的混乱」が持続するための条件を、ベイトソンのコミュニケーション理論を参考に分析する。そして、その条件が成立している相互行為を「混乱の対話」と名付ける。

\subsection{「哲学的混乱」が持続する条件}
本節では、「哲学的混乱」が持続するためには、集団内での相互行為が重要であることを示す。
「哲学的混乱」は、二通りの方法で終了する。第一に、独断で終了する。自らに課した問いに答えが見つかったという感覚が得られることで、それ以降その問いに関する「哲学的混乱」は語られなくなる。また、「哲学的混乱」は時に忘却される。それを語る意欲を人が無くし、探究の記憶が忘れられたとき、「哲学的混乱」は終了する。


このどちらも、「哲学的混乱」が一人で語られるとき、より生じやすい。
自分一人の視点から見た世界においては、独断的な解釈によって満足することができる。
そして、ある問題について長く考えていると、気づかぬうちに問題自体を自分の考えに合わせて変化させている、ということも往々にしてある。忙しい日々が続くと、それまでに考えていたことを忘れてしまうこともよくあるだろう。
このように、「哲学的混乱」は、個人内に閉じた物語であった場合には持続しづらい。


それでは、「哲学的混乱」が共有された場合はどうだろうか。第一の点については、集団の影響は、ナラティヴを収束させる向きにも、持続させる向きにも及び得る。集団極性化\footnote{引用}はよく知られた現象であり、一つのもっともらしい考えが提出されると、皆がそれを受け入れるようになることがある。一方、意見の多様性は、個々の考えに対するさまざまな問いを生み出すこともある。このようにして、「哲学的混乱」が広がっていくことも考えられる。第二の点については、共有されることで、物語が忘れられにくくなることは明らかである。また、複数人によって記憶されることで、それが改ざんされて「完結」してしまうことも無くなる。


第一の点について詳細に検討しよう。そのために、グレゴリー・ベイトソン\footnote{引用}による二者関係の分類を引用する。
ベイトソンは、物事の間の関係を「対称的」なものと「相補的」なものに二分した。
\begin{quote}
二者関係において、AとBの行動が、(AとBによって)同じものとして見られ、しかもAの行動の強まりがBを刺激して同じ(とされる)行動を強め、逆にまたBの行動がAの〝同じ〟行動を促進するようなかたちで二つが連関しているとき、それらの行動に関して両者の関係は「対称的」(symmetric)であるという。


一方、たとえば見る行為と見せる行為とが互いにフィットするように、AとBの行動が同じではないが相互にフィットするものであり、しかもAの行動の強まりがBの行動の強まりを呼ぶようなかたちで両者が連関しているとき、それらの行動に関して両者の関係は「相補的」(complementary)であるという。%p462
\end{quote}


上述の集団極性化が生じる背景には、「対称的」なテーマが入り込んでいる。すなわち、「どちらが正しいのだろう」という比較を行う相互行為が前提とされている。ある考えが別の考えに「論破」される場合にそれは顕著である。また、もっともらしい意見を聞いて黙ってしまう、あるいは考えるのをやめるという場合にも、潜在的には、復数の考えを比較してより正しい方を選ぶというディベートの構造がある。


それでは「相補的」な相互行為があればよいのかというと、それほど単純な話だとも思えない。
本節の考察はここで留めておき、これ以上については新たな視点から次節で検討しよう。
確かなのは、多様な考えを共有することと、相互に問い合うことが重要だということだ。それはよく、「対話」と呼ばれる。

\subsection{対話}
前節で分かったとおり、「哲学的混乱」が持続するためには、集団内での、何らかの、言語を介した相互行為が介在することが不可欠である。本節では、それがどのようなものなのかを明らかにし、然るべき命名を行う。その名前は「混乱の対話」というものにしたい。


劇作家の平田オリザは、対話と会話を区別して以下のように整理している。「会話」はお互いの細かい事情や来歴を知った者同士のさらなる合意形成に重きを置く、すでに知り合った者同士の楽しいお喋りのことである。「対話」は異なる価値観のすり合わせであり、差異から出発したコミュニケーションの往復に重点を置く。この説明も、用法の説明というよりはむしろ術語の定義である。術語は、特定の前提のもとで主張や議論を行うために概念を整理するものだ。
この分類に異論はない。しかし、本章の目的においては、まだ不十分である。なぜなら、本章で検討したいのは、言語を介したある種の相互行為が、参与者に対してもたらす持続的な影響だからだ。


マルティン・ブーバーの思想は、「対話の思想」と言われる。ブーバーの関心は「対話が人に及ぼす影響」であった。ブーバーは、人が世界を〈われ―それ〉〈われ―なんじ〉という二通りの方法で「語る」と主張し、二者が「互いに向かい合うこと」を根底において、人のあるべき生を論じた。本章の目的はブーバーの関心と近いが、ブーバーの神学的な議論からは本稿の最終目的である「システムの実装」に関する示唆を引き出すことは容易ではない。本章では、同様の着想から、サイバネティックス理論を応用してシステム論的なコミュニケーション理論を提示したベイトソンの議論に従って「混乱の対話」を定義したい。

\subsubsection*{ベイトソンのコミュニケーション理論\footnote{引用}}
ベイトソンのコミュニケーション理論の核をなすのは、サイバネティックス認識論と論理階型論である。以下、簡単にではあるが紹介する。\\
\textbf{サイバネティックス認識論}


この認識論によると、生きた現実にとって「存在すること」とは「行為すること」であり、行為とはシステム自身が自己を修正する自己言及的な過程である。また、システムは円環的な回路として捉えられ、その上を流れる差異が再帰的に参照される。そして、この過程を精神(mind)という。\\
\textbf{論理階型論}


論理階型論は、認知が階層的であるということを主張する。ベイトソンによると、世界の認知は四段階の学習によって成立している。ゼロ学習は、直接的な刺激の学習である(パブロフの犬)。学習Ⅰは、状況に依存したシグナルを学習する過程である。この学習により、メッセージをコンテクストに照らし合わせて解釈できるようになる。学習Ⅱは、さらなる自己の統合の過程である。「個々の状況を表すパターン」がどのような法則によって生じているのかを学習する(=世界の構造的認知過程、すなわち通常自己と言われるもの)。ベイトソンは、さらに学習Ⅲを定義した。このレベルでは、宗教的「回心」のように、自己そのものが組み替えられる。


世界に働きかけるシステムは、これら学習の段階に対応して、階層構造をなしている。(Ⅰ)行為システムは(Ⅱ)自己システムのサブシステムである。そして、自己システムをその上位で(Ⅲ)エコシステムが包摂している。一般システム論によると、部分による部分の修正は不可能である。自己システムの変容は、(a)自己システムの一部がエコシステムからのフィードバックにより変容を受ける、あるいは(b)自己システムにおけるパターンに亀裂が走り、それによってより大きな回路における学習Ⅲが生じる(自己の組み換え)という二つの形のどちらかをとる。


ベイトソンは、(a)の例として芸術鑑賞の経験を挙げている。我々の世界の認識は、無意識下に沈んだ認知枠組みに(例えば遠近法)よって支えられている。芸術作品においては、その無意識のゲシュタルトが主題化されることがある(例えばキュビズム絵画など)。そのような作品が伝える無意識のゲシュタルトについての情報は、(Ⅲ)エコシステムの回路に対してフィードバックされる。このようにして、世界の見方が「全体論」的なものに変化し、自己のより高い統合が促される\footnote{ベイトソンは、芸術のこのような作用を「精神を癒やすものとしての芸術」(corrective nature of art)と呼んだ。}。


(b)の例として挙げられているのは、アルコール中毒者が耽溺から開放されるときの体験である。アル中患者は、(時に周囲から言われ)誘惑と戦うべきだという対称的関係の中にいる。治療者は、「自らの意思でアルコールに打ち勝つことができる」という当人の無意識的な前提を打ち砕く。そして、酒には勝てない(酒に支配されるという相補的関係の中にいる)ことを自覚させる。こうして、患者の(Ⅱ)自己システムが絶望に追いやられることで、より上位のシステム(Ⅲ)における自己の組み換えが発生する。そして、患者は自己を酒との相補的な関係の中に位置づけられるようになる。アル中患者の「開放=回心\footnote{ベイトソンは、アル中患者の経験する開放を一種の神学的体験と考えていた。}」は、このように説明される。


\subsubsection*{「哲学的混乱」に向けて}
ベイトソンのいう(a)、(b)の学習による自己変容は、ある種の相互行為を通した自己変容を記述している。これらのどちらの経験も、(Ⅱ)自己システムに学習されたパターンが維持されなくなることで、自己を包摂するより大きなシステムとの「相補的」関係が生じるという事態がその契機であった。
それでは、これらの枠組みは「哲学的混乱」が持続する状況を描写できるだろうか。これには、もう少し距離があるように考えられる。第一に、「哲学的混乱」を駆動しているのは、何よりもまず、問いとの「対称的」関係である。「相補的」態度\bou{しか}存在しないのであれば、そもそも探究は行われない。
第二に、「哲学的混乱」によって生じる自己の変容は、不可逆的で断絶を伴うものではない。自己に弾力を与え、不確実な状態に対する耐性を高めさせるものである。そこで生じるのは、他者との言語を介した相互行為を経て得られた自らの「認識=存在論」についての問いを、他者の声をフラッシュバックさせながら考え追求する、という過程である。この過程は、どのようにして自己システムの変容をもたらすのだろうか。


第一の点と、先述の「対称的」な関係のみを持つ状態は独断を生むことがある、という事実が示すのは、
「哲学的混乱」が持続するためには、「対称的」関係と「相補的」関係が、同時に成立する必要があるということだ。また、学習Ⅲによる自己の組み換えは生じない。
第二の点が示すのは、「哲学的混乱」がもたらす自己の変容は、回路に情報が流れることそれ自体(学習Ⅲ)ではなく、回路間の連絡が持続することによってもたらされる、ということである。




%意見が出尽くしたところで移行する?


%それでは、この枠組みに沿って「対話」を定義できるだろうか。しかし、もう少し距離があるように考えられる。第一に、言語を介した相互行為は、「相手と同じくらい話す」という規範的な慣習において「対称的」である。相手との関係が「相補的」であるのみであれば、相互行為は維持できない。第二に、言語を介した相互行為においては、他者との関係は動的に変化する。一方が問い、もう一方が問われることもあれば、両者が自分の考えを主張し合うこともある。そして、明らかにそれらの時間平均値は意味を持たない。


%「哲学的混乱」を持続させるような(Ⅱ)自己システムの変容のあり方を定式化し、


%それを「対話」と名付けるのがここでの目的である。もし、相互行為の結果、他者の見知らぬ価値観に触れ、世界の認識が変わったとなれば、それは(a)で説明できる。また、自分の独断的な考えが否定されることで、世界が開かれたのであれば(b)である。しかし、他者との言語を介した相互行為を経て得られた自らの「認識=存在論」についての問いを、他者の声をフラッシュバックさせながら考え追求する、という過程は、明らかに自己変容の一種であるが、(a)、(b)どちらとも異なっている。


以上を踏まえると、「哲学的混乱」が維持される中で生じている相互行為は以下のように記述できる。

\begin{enumerate}
\renewcommand{\labelenumi}{\pbox<y>{(\arabic{enumi})}}
\item 「混乱の対話」において、参与者は①他者(他の参与者)との「相補的」関係+世界との「対称的」関係、あるいは②他者との「対称的」関係+世界との「相補的」関係、のどちらかの状態にあり、それらの間を動的に遷移する。このとき、
「他者」あるいは「世界」のうち(Ⅱ)自己システムと「相補的」関係にあるものが、(Ⅲ)エコシステムから流れ出し、自己システムへと連絡する。しかし、別の一方とは「対称的」関係を結んでおり、その関係が挫折するわけでは無いので、学習Ⅲによる自己の組み換えは行われない。
\item 「混乱の対話」の中で共有された主題と、「他者」「世界」と自己システムとの、\bou{部分的に「対称的」で部分的に「相補的」}な関係は、
そこで生じた自己システムとエコシステムとの連絡と共に、全体システムの回路に「焼き付け」られる。
\item そして、「対称的」な関係を含む「焼き付け」によって、現場での相互行為が終了した後も、それぞれの参与者が日常の中で主題についての追求を行うことができるようになる。それが新たな相互行為\pbox<y>{(1)}を誘発する。
\item 同時に、自己システムとエコシステムの連絡が保たれることで、
エコシステムにおいて無意識下に沈んでいた、他者との相互行為において前提とされている認知枠組みや、世界についての無意識のゲシュタルトを意識的に思考の対象にできるようになる。これによって、部分的、表面的ではなく全体論的な世界の見方が獲得され、より統合された自己システムが達成される。
\end{enumerate}




%例えばコンシャスネス・レイジング\footnote{引用}を掲げるグループが成功する条件はこの枠組みによって分析が可能である。上記の分析に従うと、そこでの語りは、社会の問題を同定し自分たちの望むあり方に変えていこうという「対称的」目的をもって、互い考えを問い合いながら慎重に議論を重ねる(「相補的」関係)、あるいは、社会における問題を直ちに解決することは難しいという「相補的」関係の認識を持ちつつも、どのようにすれば変革が可能なのかを真剣に「対称的」に議論する、という形を取る必要があると考えられる。
%逆に、グループ内で社会と他者の双方に対して「対称―対称」関係が結ばれた場合、それは運動の内部での分裂を生む可能性がある。また、「相補―相補」関係しか存在しない場合は、問題に対する関心や事態が改善されるという期待が持続することが難しくなり\footnote{一方で、「オープンダイアローグ」のように、}、そのグループには人が集まりづらくなるだろう。




%この枠組みによって説明される状況は数多くある。例えばコンシャスネス・レイジング\footnote{引用}はその一例である。定期的に行われるこのグループにおける語りにおいては、「個人的」だと思われていた(男性の女性に対する権力支配に起因する)問題が共有されることで、それらが社会的、あるいは政治的なものとして「外在化」される。

\subsection{「混乱の対話」というシステム}
前節で定義された「混乱の対話」は、「哲学的混乱」が持続する条件を指摘するものであった。では、そこで必要条件として示されている①他者(他の参与者)との「相補的」関係+世界との「対称的」関係、②他者との「対称的」関係+世界との「相補的」関係、とはどのような状態で、それらはどのように遷移するのだろうか。


まず、①が表す状況は以下のようなものである。自身が認識する世界についてのある問いに対して、少数の原理によって答えが与えられると想定する。そして、他者と異なる視点や前提に立った、多様な考えを提出するように務める。
一方、②が表す状況は以下のようなものである。「今までに出た考えから統一的な答えを得ることは不可能だ」という考えを持ちつつも、他者の考えと自分の考えを比較し、より良い答えを提出しようと自分自身で務める。


個人の中での、これらの状態間の遷移は、「混乱の対話」システムの挙動によって決定される。すなわち、多様な考えが提出されている状態では、②が目指され、意見が少数に統合されようとしている局面では①が目指される。
これらのどちらの状況でも、自己システムの内部はダブルバインドの状態に引き裂かれている。しかし、より上位の「混乱の対話」というシステムはその矛盾によって維持されるのである。
そして、「混乱の対話」システムは、一度それが形成されると、他者との現場での相互行為が終了しても、自己システムの上位に保存され、自己システムの統合を促進する。


%しかし、その形成と維持のためには十全な相互行為が現場で行われる必要がある。しかし、現実には「混乱の対話」が行われる場面は限られており、また、参加には難しさが伴う。
%次章では、この問題を解決するため、「混乱の対話」を発生させる、ロボットを用いた対話システムを提案する。

\section{「哲学的混乱」を持続させる対話システム}
%ここまでの本稿の流れをまとめる。2章では、「哲学的混乱」に導かれた探究に自己を位置づけることは、生活の質を向上させるということを示した。そして、そのような位置づけを行うセルフ・ナラティヴを「哲学的混乱」と定義した。






この点についての検討を行うために、哲学的対話がどのように頓挫するのかを改めて整理しよう。
第一に、言語コミュニケーションに齟齬が生じ、それが著しくなることで対話が継続できなくなる。
第二に、問いに対して満足な答えが見つかったという感覚が生まれることで、それ以降の探究が行われなくなる。
第三に、問い自体に関心が払われなくなることで、それ以降の探究が行われなくなる。


第一の点は、哲学的対話の成立自体に関わる障壁であるため、哲学的対話がどのように自己変容をもたらすのか、という点を検討する上では重要ではない。この点は、後に実装を考える上で考慮することになる。


\end{document}
